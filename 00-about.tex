\chapter*{Несколько слов об этом}
\addcontentsline{toc}{chapter}{Несколько слов об этом}

\begin{flushright}
{\scriptsize Наука должна быть веселая, увлекательная и простая. Таковыми же должны быть и ученые.\\
П. Л. Капица}
\end{flushright}

    Привет читатель! Если ты читаешь это, то мы тебе не завидуем. Сегодня мало
    кому придёт в голову читать брошюрку двух никому неизвестных ребят,
    посвящённую электродинамике СВЧ. Чтобы помочь читателю справиться с горем,
    которое сподвигло его к чтению этой книги, мы на пальцах и с юмором
    пробежимся по основным аспектам и особенностям миллиметровых волноводов,
    потому как больше писать тут просто не о чем. Мы постарались сделать книгу
    легко читаемой: для понимания, читатель должен быть в курсе, что такое
    \emph{частная производная}, знать, что \( i^2 = -1 \) и что
    \emph{\bfseries на ноль делить нельзя}.

    Авторы этого произведения физической мысли благодарны С****** за то, что он
    настолько крут, что авторы ему за это благодарны.

    Книга написана по мотивам лекций по курсу (барабанная дробь) электродинамики
    СВЧ нашего любимого преподавателя М. В. Г******. Желаем вам приятного
    времяпреповождения!


