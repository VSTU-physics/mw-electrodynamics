\chapter{Объёмные резонаторы}

Объёмным резонатором называют совокупность металлических и/или диэлектрических тел, внутри или вблизи которых концентрируется электромагнитное поле. Область существования поля пожно отделить от остального пространства поверхностью S, излучение энергии через которую отсутствует или незначительно.

Количество энергии, запасенной в поле резонатора зависит от частоты и вблизи некоторых частот, называемых собственными, количество запасенной энергии резко увеличивается (при прочих равных внешних условиях).

Назначение резонаторов -- запасение энергии.

По устройству резонаторы делят на открытые и закрытые. Закрытые резонаторы окружены металлическими поверхностями (кольцевые, волноводные). Открытые -- диэлектрические (волноводные, зеркальные).


\section{Колебания в резонаторах}
\[
	\Delta\vec{E} + k^2\vec{E} = 0,\quad
	\Delta\vec{H} + k^2\vec{H} = 0.
\]

Для волноводных резонаторов
\[
	\Delta E_z + k^2 E_z = 0,\quad
	\Delta H_z + k^2 H_z = 0,
\]
с граничными условиями
\[
	\left.\vec{E}_\tau\right|_S = 0.
\]
На боковых гранях
\[
	E_z = 0,\quad \pder{H_z}{n} = 0,
\]
на торцах
\[
	H_z = 0,\quad \pder{E_z}{z} = 0.
\]

Для прямоугольного резонатора:
\[
	k^2 = \pi^2\left[\left(\frac{m}{a}\right)^2+
					 \left(\frac{n}{b}\right)^2+
					 \left(\frac{p}{l}\right)^2\right],
	\omega = \frac{k}{\sqrt{\eps\mu}},
\]


Для E-колебаний
\begin{align*}
	E_z & = E_0\sin\frac{m\pi}{a}\sin\frac{n\pi}{b}\cos\frac{p\pi}{l},\\
	E_x & = -\frac{mp\pi^2}{al g^2}E_0\cos\frac{m\pi}{a}\sin\frac{n\pi}{b}\sin\frac{p\pi}{l},\\
	E_y & = -\frac{np\pi^2}{bl g^2}E_0\sin\frac{m\pi}{a}\cos\frac{n\pi}{b}\sin\frac{p\pi}{l},\\
	H_x & = \frac{i\omega\eps}{g^2}\frac{n\pi}{b}\sin\frac{m\pi}{a}\cos\frac{n\pi}{b}\cos\frac{p\pi}{l},\\
	H_y & = -\frac{i\omega\eps}{g^2}\frac{m\pi}{a}\cos\frac{m\pi}{a}\sin\frac{n\pi}{b}\cos\frac{p\pi}{l}.
\end{align*}
Низший тип таких колебаний -- \(E_{110}\).

Для H-колебаний
\begin{align*}
	E_x & = -\frac{in\pi\omega\mu}{b g^2}H_0\cos\frac{m\pi}{a}\sin\frac{n\pi}{b}\sin\frac{p\pi}{l},\\
	E_y & = -\frac{im\pi\omega\mu}{a g^2}H_0\sin\frac{m\pi}{a}\cos\frac{n\pi}{b}\sin\frac{p\pi}{l},\\
	H_z & = H_0\cos\frac{m\pi}{a}\cos\frac{n\pi}{b}\sin\frac{p\pi}{l},\\
	H_x & = -\frac{pm\pi^2}{g^2 al}H_0\sin\frac{m\pi}{a}\cos\frac{n\pi}{b}\cos\frac{p\pi}{l},\\
	H_y & = -\frac{pn\pi^2}{g^2 bl}H_0\cos\frac{m\pi}{a}\sin\frac{n\pi}{b}\cos\frac{p\pi}{l}.
\end{align*}
Низший тип таких колебаний -- \(H_{101}\).

Рассмотрим теперь круглый резонатор:
\[
	k^2 = \left(\frac{\nu_{mn}}{a}\right)^2+
					 \left(\frac{p\pi}{l}\right)^2,
	\omega = \frac{k}{\sqrt{\eps\mu}},
\]

Для E-колебаний
\begin{align*}
	E_z & = E_0 J_m(\frac{\nu_{mn}}{a}r)\Phi(m\phi)\cos\frac{p\pi}{l},\\
	E_r & = -\frac{p\pi}{lg}E_0 J_m'(\frac{\nu_{mn}}{a}r)\Phi(m\phi)\sin\frac{p\pi}{l},\\
	E_\phi & = -\frac{p\pi}{l g^2}\frac{m}{r}E_0 J_m(\frac{\nu_{mn}}{a}r)\Phi'(m\phi)\sin\frac{p\pi}{l},\\
	H_r & = \frac{i\omega\eps}{g^2}\frac{m}{r}J_m(\frac{\nu_{mn}}{a}r)\Phi'(m\phi)\cos\frac{p\pi}{l},\\
	H_\phi & = -\frac{i\omega\eps}{g}J_m'(\frac{\nu_{mn}}{a}r)\Phi(m\phi)\cos\frac{p\pi}{l}.
\end{align*}

Для H-колебаний
\begin{align*}
	E_r & = -\frac{ip\pi\omega\mu}{g^2 l}\frac{m}{r}H_0J_m(\frac{\mu_{mn}}{a}r)\Phi'(m\phi)\sin\frac{p\pi}{l},\\
	E_\phi & = -\frac{ip\pi\omega\mu}{g l}H_0J_m'(\frac{\mu_{mn}}{a}r)\Phi(m\phi)\sin\frac{p\pi}{l},\\
	H_z & = H_0 J_m(\frac{\mu_{mn}}{a}r)\Phi(m\phi)\sin\frac{p\pi}{l},\\
	H_r & = -\frac{p\pi}{gl}H_0 J_m'(\frac{\mu_{mn}}{a}r)\Phi(m\phi)\cos\frac{p\pi}{l},\\
	H_\phi & = -\frac{p\pi}{g^2l}\frac{m}{r}H_0 J_m(\frac{\mu_{mn}}{a}r)\Phi'(m\phi)\cos\frac{p\pi}{l}.
\end{align*}
Было бы неплохо всё это вывести.

\section{Добротность объёмных резонаторов}

Начнём с баланса энергии
\[
	-\oint_S\vec{\Pi}\cdot\vec{dS} =
	i\omega\int_V\left(\frac{\eps\vec{E}\cdot\vec{E}^*}{2} -
    \frac{\mu^*\vec{H}\cdot\vec{H}^*}{2}\right)\,dV +
    \frac{1}{2}\int_V\vec{j}\cdot\vec{E}^*\,dV,
\]
Для поглощаемой мощности
\[
    -\Re\oint_S\vec{\Pi}\cdot\vec{dS} =
    \omega\int_V\left(\frac{\eps''\vec{E}\cdot\vec{E}^*}{2} +
    \frac{\mu''\vec{H}\cdot\vec{H}^*}{2}\right)\,dV + \Re
    \frac{1}{2}\int_V\vec{j}\cdot\vec{E}^*\,dV,
\]
В левой части стоит мощность, излучения с обратным знаком:
\[
	\Re\oint_S\vec{\Pi}\cdot\vec{dS} = P_\text{и} = P_\Sigma + P_{me}.
\]
Для распространяющейся в резонаторе энергии
\[
    -\Im\oint_S\vec{\Pi}\cdot\vec{dS} = \omega\int_V\left(\frac{\eps'\vec{E}\cdot\vec{E}^*}{2} -
    \frac{\mu'\vec{H}\cdot\vec{H}^*}{2}\right)\,dV + \Im P_\text{вз}
\]
или
\[
    -\Im\oint_S\vec{\Pi}\cdot\vec{dS} = 2\omega\int_V\left(\average{w_E} - \average{w_H}\right)\,dV + \Im P_\text{вз}.
\]
Добротность
\[
	Q = \frac{2\pi W}{PT} = \frac{\omega W}{P},\quad \dot{W} = -\frac{\omega}{Q}W
\]
отсюда, энергия в резонаторе меняется по закону
\[
	W = W_0 e^{-\frac{\omega t}{Q}}.
\]
Для напряжённости поля
\[
	\vec{E} = \vec{E}(0)e^{-\frac{\omega t}{2Q}}e^{i\omega t}
\]
Для комплексной частоты
\[
	\omega = \omega' + i\omega'',\quad \omega' = \omega, \quad \omega'' = \frac{\omega}{2Q}.
\]
Удобнее рассмотреть
\[
	\frac{1}{Q} = \frac{P_\text{ср} + P_{me} + P_\Sigma + P_\text{вз}}{\omega W} = \frac{1}{Q_0} + \frac{1}{Q_{ex}}.
\]
При расчёте собственной добротности \(Q_0\) учитываются 3 фактора: диэлектрические, магнитные и металлические потери.
\[
	Q_{me} = 2\frac{\mu}{\mu_{me}}\sqrt{\frac{\omega\mu_{me}\sigma}{2}}
	\frac{\int_V H^2 dV}{\oint_S H_\tau^2 dS}.
\]

\begin{problem}
	Имеется достаточно длинный отрезок волновода 23 на 10. Из мего требуется сделат перестраиваемый резонатор. Определить диапазон частот перестройки и длину резонатора при невозможности возникновения высших гармоник.
\end{problem}

\[
	\omega = \pi \sqrt{\left(\frac{m}{a}\right)^2 + \left(\frac{n}{b}\right)^2 + \left(\frac{p}{l}\right)^2}.
\]

Низшим перестраиваемым типом типом будет \( H_{101} \). Чтобы оно было низшим, необходимо:
\[
	\omega_{101} < \omega_{110},\quad  l > b.
\]
Частота при этом перестраивается в диапазоне
\[
	c/2a < f < c/2\sqrt{1/a^2 + 1/b^2},
\]
\[
	6.5~\text{ГГц} < f < 16~\text{ГГц}.
\]
С другой стороны, низшая частота для \( H_{201} \) равна \( f_max = 13\text{ГГц} \). Оценим максимальный размер волновода:
\[
	l_{min} = 1/\sqrt{c^2f_{max}^2/4 - 1/a^2} = a/\sqrt{3} = 13\text{мм}.
\]
\[
	6.5~\text{ГГц} < f < 13~\text{ГГц}.
\]

\begin{problem}
	Определить отношение a/l, при котором частоты низших типов E и H колебаний совпадают.
\end{problem}

\[
	\mu_{11}^2/a^2 + \pi^2/l^2 = \nu_{01}^2/a^2, l/a = \pi/\sqrt{\nu_{01}^2 - \mu_{11}^2} = 2.
\]
\begin{problem}
Определить 5 низших частот и указать типы колебаний резонатора \(1\times2\times3\) см и для круглого a = 1см, l = 2,5 см.
\end{problem}
\begin{problem}
Определить размеры резонатора, у которого \(E_{010}\) 5 ГГц, \(H_{111}\) 3,5 ГГц.
\end{problem}
\begin{problem}
Определить добротность резонатора 34 72 100 работающего на основном типе колебания, если материал стенок -- медь, а внутреннее заполнение имеет \(\eps = 2,35\), \(\tg\delta_\eps = 1,3\cdot10^{-5}\).
\end{problem}

Основной тип колебаний \( H_{011} \) с частотой
\[
	\omega = \frac{c\pi}{\sqrt{\eps}}\sqrt{\frac{1}{b^2} + \frac{1}{l^2}}.
\]
\[
	Q_\eps = \frac{1}{\tg\delta_\eps},
\]
\[
	Q_{me} = 2\frac{\mu}{\mu_{me}}\sqrt{\frac{\omega\mu_{me}\sigma}{2}}
	\frac{\int_V H^2 dV}{\oint_S H_\tau^2 dS}.
\]
\[
	\int_V H^2 dV = H_0^2\left( 1 + \frac{b^2}{l^2} \right)\frac{abl}{4},
\]
\[
	\int_S H^2 dS = 2H_0^2\left( 1 + \frac{b^2}{l^2} \right)\frac{bl}{4} +
	2H_0^2\frac{al}{2} + 2H_0^2\frac{b^2}{l^2}\frac{ab}{2},
\]
\[
	Q_{me} = \frac{\mu}{\mu_{me}}\sqrt{\frac{\omega\mu_{me}\sigma}{2}}
	\frac{\left( 1 + \frac{b^2}{l^2} \right)\frac{abl}{4}}{\left( 1 + \frac{b^2}{l^2} \right)\frac{bl}{4} + \frac{al}{2} + \frac{b^2}{l^2}\frac{ab}{2}}.
\]

\chapter{Возбуждение волноводов и резонаторов}
Предположим, что в резонаторе задан сторонний ток \( \vec{j}^\text{ст} \), определяемый только внешним источником и не зависящий от поля в волноводе, а на поврхности есть окно связи, в котором заданы поля.

Обычно сторонний ток вводится либо электрическим зондом (штырь), либо магнитным (петля).

Граничные условия в этом случае
\begin{align*}
	& \vec{E}_\tau = 0 (S - S_0),\\
	& \vec{E}_\tau = \vec{E}_{\tau0} (S_0),\\
	& \vec{H}_\tau = \vec{H}_{\tau0} (S_0).\\
\end{align*}

\section{Возбуждение волновода}
Рассмотрим теперь волновод, в котором возбуждающие элементы сосредоточены на отрезке \([z_1, z_2]\). При этом в волноводе в обе стороны от этой области распространяются волны. Поля будем искать на достаточно большом расстоянии от источников. Поле будем искать в виде разложения по собственным волнам (\( z > z_2 \)):
\[
	\vec{E} = \sum A_m\vec{E}_m,\quad \vec{H} = \sum A_m\vec{H}_m.
\]

Лемма Лоренца:
\[
  \oint_S\left( \vec{E}_1\times\vec{H}_2 - \vec{E}_2\times\vec{H}_1 \right)\cdot d\vec{S} = \int_V \left(
  \vec{j}_1^\text{ст}\cdot\vec{E}_2 - \vec{j}_2^\text{ст}\cdot\vec{E}_1\right) dV.
\]

Рассмотрим теперь в качестве \( \vec{E}_1 \) поля в системе, а \( \vec{E}_2 \) -- поле некоторой собственной волны. Тогда
\[
  \oint_S\left( \vec{E}\times\vec{H}_m - \vec{E}_m\times\vec{H} \right)\cdot d\vec{S} = \int_V \vec{j}^\text{ст}\cdot\vec{E}_m dV.
\]

Распишем подробнее:
\begin{gather*}
  \int_{S_2}\left( \sum A_n\vec{E}_n\times\vec{H}_m - \vec{E}_m\times\sum A_n\vec{H}_n \right)\cdot\vec{z}_0 dS
  -\int_{S_2}\left( \sum A_{-n}\vec{E}_{-n}\times\vec{H}_m - \vec{E}_m\times\sum A_{-n}\vec{H}_{-n} \right)\cdot\vec{z}_0 dS +\\+
  \int_S\left( \vec{E}\times\vec{H}_m - \vec{E}_m\times\vec{H} \right)\cdot d\vec{S} = \int_V \vec{j}^\text{ст}\cdot\vec{E}_m dV.
\end{gather*}

Так как \( \vec{E}_S = 0 \),
\begin{gather*}
  \sum A_n\int_{S_2}\left(\vec{E}_n\times\vec{H}_m - \vec{E}_m\times\vec{H}_n \right)\cdot\vec{z}_0 dS
  -\sum A_{-n}\int_{S_2}\left( \vec{E}_{-n}\times\vec{H}_m - \vec{E}_m\times\vec{H}_{-n} \right)\cdot\vec{z}_0 dS +\\+
  \int_{S_0} \vec{E}_0\times\vec{H}_m \cdot d\vec{S} = \int_V \vec{j}^\text{ст}\cdot\vec{E}_m dV.
\end{gather*}

Учитывая ортогональность собственных волн для положительного \( m \) имеем

\[
  -A_{-m}\int_{S_2}\left( \vec{E}_{-m}\times\vec{H}_m - \vec{E}_m\times\vec{H}_{-m} \right)\cdot\vec{z}_0 dS =
  -\int_{S_0} \vec{E}_0\times\vec{H}_m \cdot d\vec{S} + \int_V \vec{j}^\text{ст}\cdot\vec{E}_m dV.
\]

Рассмотрим норму волны
\[
	N_m = \int \vec{E}_m \times \vec{H}_m^* \cdot d\vec{S}
\]

\[
	A_{-m} = \frac{1}{2N_m} \left[ -\int_{S_0} \vec{E}_0\times\vec{H}_m \cdot d\vec{S} + \int_V \vec{j}^\text{ст}\cdot\vec{E}_m dV \right]
\]

Для \( -m \) гармоники имеем

\[
  A_{m}\int_{S_2}\left( \vec{E}_{m}\times\vec{H}_{-m} - \vec{E}_{-m}\times\vec{H}_{m} \right)\cdot\vec{z}_0 dS =
  -\int_{S_0} \vec{E}_0\times\vec{H}_{-m} \cdot d\vec{S} + \int_V \vec{j}^\text{ст}\cdot\vec{E}_{-m} dV.
\]

\[
	A_m = \frac{1}{2N_m} \left[ -\int_{S_0} \vec{E}_0\times\vec{H}_{-m} \cdot d\vec{S} + \int_V \vec{j}^\text{ст}\cdot\vec{E}_{-m} dV \right]
\]

Итого:

\[
	A_{\pm m} = \frac{1}{2N_m} \left[ \int_V \vec{j}^\text{ст}\cdot\vec{E}_{\mp m} dV - \int_{S_0} \vec{E}_0\times\vec{H}_{\mp m} \cdot d\vec{S} \right]
\]

\section{Возбуждение резонаторов}
\[
	\vec{E} = \sum A_n \vec{E}_n,\quad \vec{H} = \sum B_n \vec{H}_n
\]
Запишем уравнения Максвелла для собственных колебаний
\[
	\begin{cases}
	\rotor \vec{E}_n = -i\omega_n\mu\vec{H}_n\\
	\rotor \vec{H}_n = i\omega_n\eps\vec{E}_n,
	\end{cases}
\]
и для картины поля вцелом
\[
	\begin{cases}
	\rotor \vec{E} = -i\omega\mu\vec{H}\\
	\rotor \vec{H} = i\omega\eps\vec{E} + \vec{j}^\text{ст}.
	\end{cases}
\]

\[
	\divergence \vec{E}_n\times\vec{H}^* = \vec{H}^* \cdot \rotor\vec{E}_n -\vec{E}_n \cdot \rotor\vec{H}^* = -i\omega_n\mu\vec{H}_n\cdot\vec{H}^* + i\omega\eps^*\vec{E}_n\vec{E}^* - {\vec{j}^\text{ст}}^*\vec{E}_n
\]

\[
	\divergence \vec{E}\times\vec{H}_n^* = \vec{H}_n^* \cdot \rotor\vec{E} -\vec{E} \cdot \rotor\vec{H}_n^* = -i\omega\mu\vec{H}\cdot\vec{H}_n^* + i\omega_n\eps^*\vec{E}\vec{E}_n^*.
\]

Интегрируя два этих выражения по объёму резонатора получаем

\[
	\oint \vec{E}_n\times\vec{H}^* \cdot d\vec{S} = 0 = -i\omega_n 2W_nB_n^* + i\omega 2W_nA_n^* + \int {\vec{j}^\text{ст}}^*\vec{E}_n dV,
\]

\[
	\oint \vec{E}\times\vec{H}_n^* \cdot d\vec{S} = \int_{S_0} \vec{E}_0\times\vec{H}_n^* \cdot d\vec{S} = -i\omega 2W_nB_n + i\omega_n 2W_nA_n.
\]

Получаем систему из двух уравнений с двумя неизвестными
\[
	\begin{cases}
		\omega A_n - \omega_n B_n = \cfrac{i}{2W_n}\int \vec{j}^\text{ст}\vec{E}_n^* dV\\
		\omega_n A_n - \omega B_n = -\cfrac{i}{2W_n} \int_{S_0} \vec{E}_0\times\vec{H}_n^* \cdot d\vec{S}.
	\end{cases}
\]
Отсюда
\begin{align*}
	& A_n = \frac{1}{2W_n(\omega^2-\omega_n^2)}\left[ \omega\int \vec{j}^\text{ст}\vec{E}_n^* dV + \omega_n\int_{S_0} \vec{E}_0\times\vec{H}_n^* \cdot d\vec{S} \right],\\
	& B_n = -\frac{1}{2W_n(\omega^2-\omega_n^2)}\left[ \omega_n\int \vec{j}^\text{ст}\vec{E}_n^* dV + \omega\int_{S_0} \vec{E}_0\times\vec{H}_n^* \cdot d\vec{S} \right].
\end{align*}
