\chapter{Распространение волн}
Рассмотрим распространение плоской волны вдоль оси \(Oz\). При этом

\[
    \pder{}{x} = \pder{}{y} = 0,
\]

\[
    \begin{cases}
        -\pder{H_y}{z} = i\omega\eps E_x,\quad \pder{H_x}{z} = i\omega\eps E_y,\\
        -\pder{E_y}{z} = -i\omega\mu H_x,\quad \pder{E_x}{z} = -i\omega\mu H_y,\\
        0 = i\omega\eps E_z,\quad 0 = -i\omega\mu H_z.
    \end{cases}
\]
Отсюда
\[
    \begin{cases}
        \ppder{E_x}{z} + k^2 E_x = 0,\\
        \ppder{H_y}{z} + k^2 H_y = 0,\\
    \end{cases}
\]
где \( k = \sqrt{\eps\mu} = \omega n / c \).
Решение ищем в виде суперпозиции волн, бегущих в разных направлениях вдоль оси
\( Oz \):

\[
    E_x = Ae^{-ikz} + Be^{ikz},\quad H_y = Z_c^{-1} (Ae^{-ikz} - Be^{ikz}),
\]

где \( Z_c = \sqrt{\mu / \eps} \) -- характеристическое сопротивление среды.
Далее будем рассматривать только падающую волну:
\[
    E_x = Ae^{-ikz},\quad H_y = Z_c^{-1} Ae^{-ikz}.
\]
В непоглощающей среде \(\eps\) и \(\mu\) -- вещественные, поэтому и \(Z_c\)
также вещественно.

С учётом временной зависимости:
\[
    E_x = Ae^{-i(kz-\omega t)},\quad H_y = Z_c^{-1} Ae^{-i(kz-\omega t)}.
\]

Фазовая скорость волны определяется из всем известной формулы:
\[
v_f=\frac{dz}{dt}=\frac{\omega}{k}=\frac{c}{n_s}=u,
\]
из которой получается, что фазовая скорость численно равна скорости света в среде.

Групповая скорость волны определяется из не менее известной формулы:
\[
v_g=\frac{d\omega}{dk}=v_f - \lambda_s\frac{dv_f}{d\lambda_s},
\]

где $\lambda_s$ --- длина волны в среде.


Последнее соотношение носит гордое имя формулы Рэлея.

\section{Случай с поглощающей средой}

Поглощающая среда подобно ужасному упырю из страшных сказок отнимает жизненные силы у главного героя
нашего повествования --- маленькой, но 
крайне отважной электромагнитной волны (ЭМВ). С каждой пройденной единицей пути у волны остаётся всё меньше и меньше
ей так необходимой энергии. В связи с этим диэлектрическая и магнитная проницаемости среды начинают комплексовать:

$$\eps \to \dot{\eps} = \eps'-i\eps'', \; \mu \to \dot{\mu}. $$

Что неизбежно приводит к появлению комплексов у волнового числа $k \to \dot{k}$.


Не бойся, читатель! Мы будем рассматривать случай слабого затухания ($\eps''<<\eps', \mu''<< \mu'$), когда 
 у ЭМВ есть шанс однажды вырваться в родной вакуум из пут этой коварной среды.


Тогда выражение для k 

$$\dot{k}=\omega \sqrt{(\eps'-i\eps'')(\mu'-i\mu'')} $$

можно приближённо записать

$$\dot{k} \approx \omega \sqrt{\eps'\mu'}(1-\frac{1}{2}(tg(\delta_{\eps})+tg(\delta_{\mu})))=k'-ik'', $$
где k'' описывает затухание волны.

Сопротивление среды 

$$Z_c=\sqrt{\frac{\mu'-i\mu''}{\eps'-i\eps''}} 
\approx \sqrt{\frac{\mu'}{\eps'}}(1+\frac{1}{2}(tg(\delta_{\eps})-tg(\delta_{\mu}))). $$

Видно, что из-за $(tg(\delta_{\eps})+tg(\delta_{\mu}))$ электрические и магнитные
потери суммируются. Кроме того, волну так корёжит в данной среде, что электрические и магнитные колебания
перестают быть синфазными при $tg(\delta_{\eps}) \ne tg(\delta_{\mu})$.

\section{Случай с неферромагнитной средой}

Данная среда является в некотором смысле попсовой, т.к. из неё в осномном и делают различные 
крутые штуки, в частности, волноводы. Типичным представителем данной категории сред является медь, 
проводимость (внимательный исследователь нашего чтива сообразит, что речь идёт об удельной проводимости)
[кстати, почему?] которой (только задумайтесь!) имеет 7 порядок [тут сразу вопрос к читателю:
в каких единицах измеряется проводимость в СИ? В СГС?].


Итак, для неферромагнитной среды имеем

$$\mu=\mu_0, \; \dot{\eps}=\eps_0-i\frac{\sigma}{\omega}, $$

с учетом того, что порядок частот обычно в районе 11, мы вправе записать

$$\frac{\sigma}{\omega} \sim 10^{-3} >> 10^{-11} \sim \eps_0. $$

Модуль коэффициента приломления среды (можно смело писать металла)

$$\abs{n_{met}}=\abs{\sqrt{\dot{\eps_r}\dot{\mu_r}}}=\abs{\sqrt{1-i\frac{\sigma}{\omega\eps_0}}} \approx 
\sqrt{\frac{\sigma}{\omega\eps_0}} >> 1.$$

Тут мы сделаали финт ушами, когда применили известную из ТФКП формулу $\abs{\sqrt{z}}=\sqrt{\abs{z}}$.

Т.к. $n_{met}=\frac{\sin(\alpha)}{\sin(\betta)}$, то падающая волна будет распространятся перпендикулярно
поверхности металла после встречи с ним.
