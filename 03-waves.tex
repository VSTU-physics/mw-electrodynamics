\chapter{Распространение волн}
Рассмотрим распространение плоской волны вдоль оси \(Oz\). При этом

\[
    \pder{}{x} = \pder{}{y} = 0,
\]

\[
    \begin{cases}
        -\pder{H_y}{z} = i\omega\eps E_x,\quad \pder{H_x}{z} = i\omega\eps E_y,\\
        -\pder{E_y}{z} = -i\omega\mu H_x,\quad \pder{E_x}{z} = -i\omega\mu H_y,\\
        0 = i\omega\eps E_z,\quad 0 = -i\omega\mu H_z.
    \end{cases}
\]
Отсюда
\[
    \begin{cases}
        \ppder{E_x}{z} + k^2 E_x = 0,\\
        \ppder{H_y}{z} + k^2 H_y = 0,\\
    \end{cases}
\]
где \( k = \sqrt{\eps\mu} = \omega n / c \).
Решение ищем в виде суперпозиции волн, бегущих в разных направлениях вдоль оси
\( Oz \):

\[
    E_x = Ae^{-ikz} + Be^{ikz},\quad H_y = Z_c^{-1} (Ae^{-ikz} - Be^{ikz}),
\]

где \( Z_c = \sqrt{\mu / \eps} \) -- характеристическое сопротивление среды.
Далее будем рассматривать только падающую волну:
\[
    E_x = Ae^{-ikz},\quad H_y = Z_c^{-1} Ae^{-ikz}.
\]
В непоглощающей среде \(\eps\) и \(\mu\) -- вещественные, поэтому и \(Z_c\)
также вещественно.

С учётом временной зависимости:
\[
    E_x = Ae^{-i(kz-\omega t)},\quad H_y = Z_c^{-1} Ae^{-i(kz-\omega t)}.
\]

Фазовая скорость волны определяется из всем известной формулы:
\[
v_f=\frac{dz}{dt}=\frac{\omega}{k}=\frac{c}{n_s}=u,
\]
из которой получается, что фазовая скорость численно равна скорости света в среде.

Групповая скорость волны определяется из не менее известной формулы:
\[
v_g=\frac{d\omega}{dk}=v_f - \lambda_s\frac{dv_f}{d\lambda_s},
\]

где $\lambda_s$ --- длина волны в среде.


Последнее соотношение носит гордое имя формулы Рэлея.
