\chapter{Волны в волноводе}
\section{В прямоугольном волноводе}

\begin{gather*}
	E_x = \frac{-i}{g^2}\left(h\pder{E_z}{x} + \omega\mu\pder{H_z}{y}\right),\\
	E_y = \frac{-i}{g^2}\left(h\pder{E_z}{y} - \omega\mu\pder{H_z}{x}\right),\\
	H_x = \frac{i}{g^2}\left(\omega\eps\pder{E_z}{y} - h\pder{H_z}{x}\right),\\
	H_y = \frac{-i}{g^2}\left(\omega\eps\pder{E_z}{x} + h\pder{H_z}{y}\right).
\end{gather*}

\section{В круглом волноводе}
\begin{gather*}
	E_r =    \frac{-i}{g^2}\left(h\pder{E_z}{r} + \frac{\omega\mu}{r}\pder{H_z}{\phi}\right),\\
	E_\phi = \frac{-i}{g^2}\left(\frac{h}{r}\pder{E_z}{\phi} - \omega\mu\pder{H_z}{r}\right),\\
	H_r =    \frac{i}{g^2}\left(\frac{\omega\eps}{r}\pder{E_z}{\phi} - h\pder{H_z}{r}\right),\\
	H_\phi = \frac{-i}{g^2}\left(\omega\eps\pder{E_z}{r} + \frac{h}{r}\pder{H_z}{\phi}\right).
\end{gather*}

Из уравнения Гельмгольца получаем для продольных компонент поля
\begin{gather}
	\Delta_\perp E_z + g^2 E_z = 0,\\
	\Delta_\perp H_z + g^2 H_z = 0.
\end{gather}

В системе могут существовать отдельные TE и TM волны, если на контуре волновода
\[
	\pder{E_z}{l} = \pder{H_z}{l} = 0.
\]
Если условие не выполняются, то в системе будут наблюдаться гибридные волны.

\section{Прямоугольном волновод}

Рассмотрим прямоугольный волновод \( a \times b, a > b, a || Ox \).

E-волны (TM) имеют в нём следующий вид:

\begin{align*}
	& E_z = E_0\sin\frac{m\pi x}{a}\sin\frac{n\pi y}{b},\\
	& H_z = 0,\\
	& E_x = -i\frac{hm\pi}{g_{m,n}^2a}E_0\cos\frac{m\pi x}{a}\sin\frac{n\pi y}{b},\\
	& E_y = -i\frac{hn\pi}{g_{m,n}^2b}E_0\sin\frac{m\pi x}{a}\cos\frac{n\pi y}{b},\\
	& H_x = -\frac{\omega\eps}{h}E_y = i\frac{\omega\eps n\pi}{g_{m,n}^2b}E_0
	 								\sin\frac{m\pi x}{a}\cos\frac{n\pi y}{b},\\
	& H_y = \frac{\omega\eps}{h}E_x = -i\frac{\omega\eps m\pi}{g_{m,n}^2a}E_0
	 								\cos\frac{m\pi x}{a}\sin\frac{n\pi y}{b}.
\end{align*}

H-волны (TE):
\begin{align*}
	& E_z = 0,\\
	& H_z = H_0\cos\frac{m\pi x}{a}\cos\frac{n\pi y}{b},\\
	& E_x = i\frac{\omega\mu n\pi}{g_{m,n}^2b}H_0\cos\frac{m\pi x}{a}\sin\frac{n\pi y}{b},\\
	& E_y = -i\frac{\omega\mu m\pi}{g_{m,n}^2a}H_0\sin\frac{m\pi x}{a}\cos\frac{n\pi y}{b},\\
	& H_x = -\frac{h}{\omega\mu}E_y = i\frac{hm\pi}{g_{m,n}^2a}H_0
	 								\sin\frac{m\pi x}{a}\cos\frac{n\pi y}{b},\\
	& H_y = \frac{h}{\omega\mu}E_x = -i\frac{h n\pi}{g_{m,n}^2b}H_0
	 								\cos\frac{m\pi x}{a}\sin\frac{n\pi y}{b}.
\end{align*}

Здесь \( g_{m,n} \) -- поперечное волновое число, определяемое выражением
\[
	g_{m,n} = \pi\sqrt{\frac{m^2}{a^2} + \frac{n^2}{b^2}}.
\]

Так как
\[
	h^2 = \beta^2 - g_{m,n}^2,
\]
то для данного типа волны существует критическая частота, ниже которой эта волна возбуждаться не может. Она называется критической:
\[
	\omega_c = \frac{c}{\sqrt{\eps_r \mu_r}}g_{m,n}.
\]

Дисперсионное соотношение имеет вид:
\[
	h^2 = \frac{\omega^2 \eps_r \mu_r}{c^2} - g_{m,n}^2,
\]
откуда
\[
	v_p = \frac{\omega}{h} = \frac{c}{\sqrt{\eps_r\mu_r}}\frac{1}{\sqrt{1 - \omega_c^2 / \omega^2}},
\]

\[
	v_g = \der{\omega}{h} = \frac{c}{\sqrt{\eps_r\mu_r}}\sqrt{1 - \omega_c^2 / \omega^2}.
\]

Заметим, что
\[
	v_p v_g = \frac{c^2}{\eps_r\mu_r} \text{ --- квадрат скорости света в среде.}
\]

\begin{center}
\begin{tikzpicture}
\begin{axis}[
	xmin = 0, xmax = 1, ymin = 0, ymax = 4,
    xlabel = {$\lambda_0 / \lambda_c$},
    ylabel = {$v / c$},
    minor tick num = 2
]
\addplot[domain=0:0.97,samples=100]{(1 - x^2)^-0.5} node[pos=0.75,pin=170:{$v_p$}]{};
\addplot[domain=0:1,samples=100]{(1 - x^2)^0.5} node[pos=0.8,pin=100:{$v_g$}]{};
\end{axis}
\end{tikzpicture}
\end{center}