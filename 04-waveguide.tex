\chapter{Волны в ограниченной среде}

Волновод -- это конструкция, которая позволяет волне переносить энергию внутри себя. С помощью неё можно протащить ЭМВ от их источника к приёмнику не рассеивая их попусту, а также делать другие странные вещи. Как правило волноводы бывают металлические (с металлической границей и диэлектрическим заполнением) и диэлектрические (как металлические, только без металла).

Рассмотрим простейший вариант волновода, а именно объём, ограниченный цилиндрической поверхностью. Направим ось \( Oz \) вдоль оси волновода. Тогда поля в волноводе будут иметь вид
\[
	\vec{E}(\vec{r}, t) = \vec{E}(\vec{r}_\perp)e^{i(\omega t - hz)},\quad
	\vec{H}(\vec{r}, t) = \vec{H}(\vec{r}_\perp)e^{i(\omega t - hz)}
\]
где \( h \) -- волноводное (продольное) волновое число.

Волна в волноводе удовлетворяет однородному волновому уравнению:
\[
	\square \vec{E}(\vec{r}, t) = 0,\quad \square = \Delta - \frac{1}{u^2}\ppder{}{t}.
\]
Подставляя вид поля, получаем
\[
	\left(\Delta_\perp - h^2 + \frac{\omega^2}{u^2}\right) \vec{E} = 0,
\]
где под \( \vec{E} \) понимается поле, зависящее только от поперечных координат. Везде далее в этой главе это будет подразумеваться. \( u \) -- скорость света в среде, заполняющей волновод.

Обозначим \( k = \omega / u \) -- постоянная распространения, \( g = \sqrt{k^2 - h^2} \) -- поперечное волновое число. Тогда уравнение примет вид уравнения Гельмгольца со всеми вытекающими решениями:
\[
	(\Delta_\perp + g^2) \vec{E} = 0, \quad (\Delta_\perp + g^2) \vec{H} = 0
\]
Вид этих решений зависит от выбора системы координат.

Отметим, что компоненты связаны друг с другом через уравнения Максвелла:
\[
	\begin{cases}
		\rotor\vec{E} = -i\omega\mu\vec{H},\\
		\rotor\vec{H} = i\omega\eps\vec{E},
	\end{cases}
\]

Выберем в каждой точке перпендикулярные единичные вектора \( \vec{e}_\xi \) и \( \vec{e}_\eta \) так, чтобы \( (\vec{e}_\xi, \vec{e}_\eta, \vec{e}_z) = 1 \) и введём вдоль них координаты \( \xi \) и \( \eta \). Это соответствует любым ортогональным координатам. Расписывая покоординатно, имеем
\[
	\begin{cases}
		\frac{1}{l_\eta}\pder{E_z}{\eta} + ih E_\eta = -i\omega\mu H_\xi,\\
		-\frac{1}{l_\xi}\pder{E_z}{\xi} - ih E_\xi = -i\omega\mu H_\eta,\\
		\frac{1}{l_\eta}\pder{H_z}{\eta} + ihH_\eta = i\omega\eps E_\xi,\\
		-\frac{1}{l_\xi}\pder{H_z}{\xi} - ihH_\xi = i\omega\eps E_\eta,\\
	\end{cases}
\]
где \( l_\alpha \) -- коэффициенты Ламэ (буква \( H \) занята магнитным полем).

Можно рассматривать поперечные компоненты как функции продольных:
\[
	\begin{cases}
		ihE_\eta + i\omega\mu H_\xi = -\frac{1}{l_\eta}\pder{E_z}{\eta},\\
		i\omega\eps E_\eta + ihH_\xi = -\frac{1}{l_\xi}\pder{H_z}{\xi},\\
		-ihE_\xi + i\omega\mu H_\eta = \frac{1}{l_\xi}\pder{E_z}{\xi},\\
		- i\omega\eps E_\xi + ihH_\eta = -\frac{1}{l_\eta}\pder{H_z}{\eta},\\
	\end{cases}
\]
откуда
\[
	\begin{cases}
		E_\xi  = \frac{-i}{g^2}\left(\frac{h}{l_\xi}\pder{E_z}{\xi} +\frac{\omega\mu}{l_\eta}\pder{H_z}{\eta}\right),\\
		E_\eta = \frac{-i}{g^2}\left(\frac{h}{l_\eta}\pder{E_z}{\eta} - \frac{\omega\mu}{l_\xi}\pder{H_z}{\xi}\right),\\
		H_\xi  = \frac{i}{g^2}\left(\frac{\omega\eps}{l_\eta}\pder{E_z}{\eta} - \frac{h}{l_\xi}\pder{H_z}{\xi}\right),\\
		H_\eta = \frac{-i}{g^2}\left(\frac{\omega\eps}{l_\xi}\pder{E_z}{\xi} + \frac{h}{l_\eta}\pder{H_z}{\eta}\right),
	\end{cases}
\]
а \( E_z \) и \( H_z \) -- решения уравнений Гельмгольца:
\[
	(\Delta_\perp + g^2) E_z = 0, \quad (\Delta_\perp + g^2) H_z = 0
\]

Для полной постановки задачи осталось определить граничные условия.
