\chapter{Основные уравнения}
\section{Уравнения Максвелла}
  Жили-были электрическое и магнитное поле в средах различных. И изучали их мужи
  заморские. И установили они законы диковинные. И связал Макссвелл поля эти
  узами уравнений своих. И переписал потом Хевисайд уравнения эти в форме
  удобочитаемой:

  \[
    \begin{cases}
      \divergence\vec{B} = 0,\\
      \divergence\vec{D} = \rho,\\
      \rotor\vec{H} = \vec{j} + \pder{\vec{D}}{t},\\
      \rotor\vec{E} = -\pder{\vec{B}}{t}.
    \end{cases}
  \]

  Создаются поля эти токами да зарядами, но не ими едиными. Могут поля друг
  друга порождать в изменениях своих.

  Символы басурманские в правой части расшифровать надобно. Плотность тока
  \[
      \vec{j} = \vec{j}^\text{пр} + \vec{j}^\text{к} + \vec{j}^\text{ст},
  \]
  из частей состоит: тока проводимости, ковекционного и стороннего, окаянного.
  Токи проводимости и конвекционный соотношениями связанны с полем и плотностью
  зарядовой:
  \[
      \vec{j}^\text{пр} = \sigma\vec{E}, \vec{j}^\text{к} = \rho\vec{v}.
  \]

  Индукции же \( \vec{D} \) и \( \vec{B} \) с напряжённостями \( \vec{E} \) и
  \( \vec{H} \) связаны уравнениями материальными.

\section{Материальные уравнения}
  Уравнения эти вид имеют следующий:
  \[
      \vec{D} = \eps\vec{E},\quad
      \vec{B} = \mu\vec{H}.
  \]
    
  \textit{Хозяюшке на заметку: чтобы узнать, проводник перед Вами или изолятор,
  спросите, сколько у него \(\omega\eps/\sigma\). Если 10 или больше ---
  то это изолятор. У проводника эта величина существенно меньше 1 и он скорее
  промолчит.}

\section{Граничные условия}
  Если поля сбежать пытаются из родной среду в чужеземную, то на границе
  происходит казус с их преломлением, следующий непосредственно из уравнений
  Максвелла:
  \[
    \vec{n}\times(\vec{H}_1 - \vec{H}_2) = \vec{I}_s,
  \]
  \[
    \vec{n}\times(\vec{E}_1 - \vec{E}_2) = 0,
  \]
  \[
    \vec{n}\cdot(\vec{D}_1 - \vec{D}_2) = \rho_s,
  \]
  \[
    \vec{n}\cdot(\vec{B}_1 - \vec{B}_2) = 0,
  \]
  где \(\vec{n}\) --- вектор нормальный единичный, в первую среду из границы
  впивающийся.

\section{Закон сохранения энергии}
  Уравнения Максвелла принуждают поля электрические да магнитные жить по закону
  Божьему сохранения энергии:
  \[
    \pder{}{t}\left(\frac{\vec{E}\cdot\vec{D}}{2} +
    \frac{\vec{H}\cdot\vec{B}}{2}\right) =
    -\divergence(\vec{E}\times\vec{H}) - \vec{j}\cdot\vec{E},
    \quad\textit{(в образе дифференциальном)}
  \]
  \[
    \pder{}{t}\int_V\left(\frac{\vec{E}\cdot\vec{D}}{2} +
    \frac{\vec{H}\cdot\vec{B}}{2}\right)\,dV = -\oint_S\vec{\Pi}\cdot\vec{dS}
    - \int_V\vec{j}\cdot\vec{E}\,dV.
    \quad\textit{(в величии интегральном)}
  \]

  Вектор \( \vec{\Pi} = \vec{E}\times\vec{H} \) --- вектор Умова-Пойнтинга,
  который языки иноземные вектором Пойнтинга величают (санкции?).

\section{Гармонические колебания и комплексные амплитуды}
  Обычно поля непостоянны бывают, колеблются. Разложим их, подобно иноземцу
  прозорливому на осцилляции гармонические. И опишем их амплитудами
  комплексными, ведь подобно человеку, у каждого гармонически колеблющегося поля
  есть друг воображаемый, сдвинутый по фазе на \( \pi/2 \):
  \[
    \vec{E}(\vec{r}, t) = \Re \left(\vec{E}(\vec{r})e^{i\omega t} \right)
  \]
  Используя друга этого заменить оператор \(\pder{}{t} = i\omega \)
  позволительно:
  \[
    \begin{cases}
      \divergence \vec{B} = 0\\
      \divergence\vec{D} = \rho\\
      \rotor\vec{H} = \vec{j}^\text{ст}+\sigma\vec{E} + i\omega\vec{D}\\
      \rotor\vec{E} = -i\omega\vec{B}.
    \end{cases}
  \]
  В материальных уравнениях индукции сдвинутся по фазе от напряжённостей
  \[
    \vec{D} = \eps e^{i\Delta_\eps}\vec{E},\quad
    \vec{B} = \mu e^{i\delta_\mu}\vec{H}.
  \]
  Если подставить их в уравнения Максвелла, то:
  \[
  \begin{cases}
    \divergence\mu e^{i\delta_\mu}\vec{H} = 0,\\
    \divergence\eps e^{i\Delta_\eps}\vec{E} = \rho,\\
    \rotor\vec{H} = \vec{j}^\text{ст} + i\omega\eps e^{i\Delta_\eps}\vec{E},\\
    \rotor\vec{E} = -i\omega\mu e^{i\delta_\mu}\vec{H},
  \end{cases}
  \]
  повод появится ввести комплексные проницаемости:
  \[
    \eps = \eps' - i\eps'', \mu = \mu' - i\mu'',
  \]
  и связать их со сдвигами фазовыми из уравнений материальных:
  \[
    \eps' = \eps\cos\Delta_\eps,\quad\eps'' = \eps\sin\Delta_\eps +
    \frac{\sigma}{\omega},
  \]
  \[
    \mu' = \eps\cos\delta_\mu,\quad\mu'' = \eps\sin\delta_\mu.
  \]

  Так как поля в колебаниях гармоничных пребывают, то о средних рассуждать
  полезно. Так как \( \average{\cos^2{\omega t}} = 1/2 \), то
  средние плотности энергии:
  \[
    \average{w_E} = \frac{1}{4}\eps'\vec{E}\vec{E}^*,\quad
    \average{w_H} = \frac{1}{4}\mu'\vec{H}\vec{H}^*,
  \]
  а вектор Пойнтинга комплексный смысл среднего уже имеет:
  \[
    \vec{\Pi} = \frac{1}{2}\vec{E}\times\vec{H}^*.
  \]
  Учитывая это, вспомним про закон Божий сохранения энергии:
  \[
    i\omega\int_V\left(\frac{\eps\vec{E}\cdot\vec{E}^*}{2} -
    \frac{\mu^*\vec{H}\cdot\vec{H}^*}{2}\right)\,dV =
    -\oint_S\vec{\Pi}\cdot\vec{dS} -\frac{1}{2}\int_V\vec{j}\cdot\vec{E}^*\,dV.
  \]
  Забавы ради, выделим части действительные и мнимые и получим балансы мощности
  активной и реактивной:
  \[
    \omega\int_V\left(\frac{\eps''\vec{E}\cdot\vec{E}^*}{2} +
    \frac{\mu''\vec{H}\cdot\vec{H}^*}{2}\right)\,dV =
    -\Re\oint_S\vec{\Pi}\cdot\vec{dS} - \Re P_\text{вз},
  \]
  \[
    \omega\int_V\left(\frac{\eps'\vec{E}\cdot\vec{E}^*}{2} -
    \frac{\mu'\vec{H}\cdot\vec{H}^*}{2}\right)\,dV =
    -\Im\oint_S\vec{\Pi}\cdot\vec{dS} - \Im P_\text{вз}.
  \]

\section{Лемма Лоренца и теорема о взаимности}
  Лемма Лоренца:
  \begin{quote}
    если есть два независимых тока, создающих по отдельности соответствующие
    поля, то
    \[
      \divergence\{ \vec{E}_1\times\vec{H}_2 - \vec{E}_2\times\vec{H}_1 \} =
      \vec{j}_1^\text{ст}\cdot\vec{E}_2 - \vec{j}_2^\text{ст}\cdot\vec{E}_1.
    \]
  \end{quote}
  
  Это очевидно, т.к
  \[
    \divergence{ \vec{E}_1\times\vec{H}_2 } = \vec{H}_2\cdot\rotor\vec{E}_1 -
    \vec{E}_1\cdot\rotor\vec{H}_2.
  \]

  Теорема о взаимности:
  \[
    \int_{V_1} \vec{j}_1^\text{ст}\cdot\vec{E}_2\,dV =
    \int_{V_2} \vec{j}_2^\text{ст}\cdot\vec{E}_1\,dV.
  \]
  Эта теорема справедлива только в средах с симметричными тензорами и утверждает,
  что от перемены мест источника и приёмника качество сигнала не меняется.
  То есть у них это взаимно.

\section{Задачи электродинамики}
  Ну и наконец, зачем нужна электродинамика:
  \begin{itemize}
  \item внутренняя задача: объем поля ограничен замкнутой поверхностью, вне
  которого поля не существует; решение существует и единственно если:
  \begin{enumerate}
  \item если среда поглощающая или регенеративная;
  \item на поверхности заданы касательные составляющие электрического или
  магнитного полей;
  \end{enumerate}
  \item внешняя задача: ищется решение для неограниченного пространства при
  наличии заданных источников поля и, возможно, областей, где поле отсутствует;
  решение существует и единственно, если:
    \begin{enumerate}
  \item если среда поглощающая;
  \item на поверхности областей, вне которых определяется поле, заданы
  касательные составляющие электрического или магнитного полей;
  \item \(\ds
        \lim_{r\to\infty}\oint_{4\pi r^2} \vec{E}\times\vec{H}^*\cdot\vec{dS}\).
  \end{enumerate}
  \end{itemize}
