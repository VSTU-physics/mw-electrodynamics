\chapter{Основные уравнения}
\section{Уравнения Максвелла}
    В основе электродинамики лежат уравнения Максвелла -- 3 экспериментальных
    закона (Кулона, Био-Савара-Лапласа, Фарадея) и факт отсутствия магнитных
    монополей, которые были облачены в не очень удобную математическую форму
    Максвеллом. А потом Хевисайд придумал для них не просто удобную, а краткую и
    красивую запись, в котором они дошли до нас:

    \[
      \begin{cases}
        \divergence \vec{B} = 0\\
        \divergence\vec{D} = \rho\\
        \rotor\vec{H} = \vec{j} + \pder{\vec{D}}{t}\\
        \rotor\vec{E} = -\pder{\vec{B}}{t},
      \end{cases}
    \]

    Если читатель не в курсе, что означают эти буквы, то советуем ему перестать
    это читать, собрать рюкзак и не прогуливать больше уроки!

    Там в правой части фигурирует плотность тока, которая в общем случае состоит
    из тока проводимости, конвекционного и стороннего тока:
    \[
        \vec{j} = \vec{j}^\text{пр} + \vec{j}^\text{к} + \vec{j}^\text{ст},
    \]
    \[
        \vec{j}^\text{пр} = \sigma\vec{E}, \vec{j}^\text{к} = \rho\vec{v}.
    \]

\section{Материальные уравнения}
    Индукции связаны с напряжённостями материальными уравнениями
    \[
        \vec{D} = \eps\vec{E},\quad
        \vec{B} = \mu\vec{H}.
    \]
    Проницаемости здесь абсолютные. Любителям СГС должно быть всё равно.
    Профессионалы СГС до сих пор не дочитают, так как они уже заметили,
    что уравнения Максвелла записаны неверно.

    \textit{На заметку: если Вы не совсем уверены, проводник перед Вами или нет, то есть
    простой способ проверить. Спросите, сколько у него \(\omega\eps/\sigma\).
    Если порядка 10 или больше --- то это изолятор. У проводников эта величина
    существенно меньше 1 и они обычно не афишируют это.}

    Плавно перейдём к ситуации, когда есть две контактирующие однородные среды
    и посмотрим, что там происходит с полями.

\section{Граничные условия}
    Если две среды контактируют друг с другом, то должны выполняться граничные
    условия, следующие непосредственно из уравнений Максвелла:
    \[
      \vec{n}\times(\vec{H}_1 - \vec{H}_2) = \vec{I}_s,
    \]
    \[
      \vec{n}\times(\vec{E}_1 - \vec{E}_2) = 0,
    \]
    \[
      \vec{n}\cdot(\vec{D}_1 - \vec{D}_2) = \rho_s,
    \]
    \[
      \vec{n}\cdot(\vec{B}_1 - \vec{B}_2) = 0.
    \]
    Нормаль \(\vec{n}\) торчит из границы в первую среду.

\section{Закон сохранения энергии}
    Из уравнений Максвелла следует закон сохранения энергии:
    \[
      \pder{}{t}\left(\frac{\vec{E}\cdot\vec{D}}{2} +
      \frac{\vec{H}\cdot\vec{B}}{2}\right) =
      -\divergence(\vec{E}\times\vec{H}) - \vec{j}\cdot\vec{E},
      \quad\textit{(в профиль)}
    \]
    \[
      \pder{}{t}\int_V\left(\frac{\vec{E}\cdot\vec{D}}{2} +
      \frac{\vec{H}\cdot\vec{B}}{2}\right)\,dV = -\oint_S\vec{\Pi}\cdot\vec{dS}
      - \int_V\vec{j}\cdot\vec{E}\,dV.
      \quad\textit{(в анфас)}
    \]

    Вектор \( \vec{\Pi} = \vec{E}\times\vec{H} \) --- вектор Умова-Пойнтинга,
    который обычно просто называют вектором Пойнтинга (санкции?).

\section{Гармонические колебания и комплексные амплитуды}
  Перейдём к полям, меняющимся по гармоническому закону. Их удобнее описывать на
  языке комплексных амплитуд. Если Вы внимательно читали минимальные необходимые
  требования к читателю, то Вас это не смутит такой переход:
  \[
    \vec{E}(\vec{r}, t) = \Re \left(\vec{E}(\vec{r})e^{i\omega t} \right)
  \]
  Используя этот аппарат можно заменить оператор \(\pder{}{t} = i\omega \)
  и переписать уравнения Максвелла (для среды, в которой нет конвекционного
  тока):
  \[
    \begin{cases}
      \divergence \vec{B} = 0\\
      \divergence\vec{D} = \rho\\
      \rotor\vec{H} = \vec{j}^\text{ст}+\sigma\vec{E} + i\omega\vec{D}\\
      \rotor\vec{E} = -i\omega\vec{B}.
    \end{cases}
  \]
  Материальные уравнения станут связывать комплексные амплитуды
  \[
    \vec{D} = \eps e^{i\Delta_\eps}\vec{E},\quad
    \vec{B} = \mu e^{i\delta_\mu}\vec{H}.
  \]
  Если подставить их в уравнения Максвелла, то интереснее, увы, не станет:
  \[
  \begin{cases}
    \divergence\mu\vec{H} = 0\\
    \divergence\eps\vec{E} = \rho\\
    \rotor\vec{H} = \vec{j}^\text{ст} + i\omega\eps\vec{E}\\
    \rotor\vec{E} = -i\omega\mu\vec{H}.
  \end{cases}
  \]
  Зато теперь есть повод ввести комплексные проницаемости:
  \[
    \eps = \eps' - i\eps'', \mu = \mu' - i\mu''.
  \]
  И связать их с теми странными параметрами из материальных уравнений:
  \[
    \eps' = \eps\cos\Delta_\eps,\quad\eps'' = \eps\sin\Delta_\eps +
    \frac{\sigma}{\omega},
  \]
  \[
    \mu' = \eps\cos\delta_\mu,\quad\mu'' = \eps\sin\delta_\mu.
  \]

  Поговорим теперь о средних. Так как \( \average{\cos^2{\omega t}} = 1/2 \), то
  средние плотности энергии:
  \[
    \average{w_E} = \frac{1}{4}\eps'\vec{E}\vec{E}^*,\quad
    \average{w_H} = \frac{1}{4}\mu'\vec{H}\vec{H}^*.
  \]
  Комплексный вектор Пойнтинга уже имеет смысл среднего:
  \[
    \vec{\Pi} = \frac{1}{2}\vec{E}\times\vec{H}^*.
  \]
  Учитывая это, вспомним про закон сохранения энергии:
  \[
    i\omega\int_V\left(\frac{\eps\vec{E}\cdot\vec{E}^*}{2} -
    \frac{\mu^*\vec{H}\cdot\vec{H}^*}{2}\right)\,dV =
    -\oint_S\vec{\Pi}\cdot\vec{dS} -\frac{1}{2}\int_V\vec{j}\cdot\vec{E}^*\,dV.
  \]
  Для развлечения выделим действительные и мнимые части и получим балансы
  активной и реактивной мощности:
  \[
    -\omega\int_V\left(\frac{\eps''\vec{E}\cdot\vec{E}^*}{2} +
    \frac{\mu''\vec{H}\cdot\vec{H}^*}{2}\right)\,dV =
    \Re\oint_S\vec{\Pi}\cdot\vec{dS} + \Re P_\text{вз},
  \]
  \[
    \omega\int_V\left(\frac{\eps'\vec{E}\cdot\vec{E}^*}{2} -
    \frac{\mu'\vec{H}\cdot\vec{H}^*}{2}\right)\,dV =
    -\Im\oint_S\vec{\Pi}\cdot\vec{dS} - \Im P_\text{вз}.
  \]

\section{Лемма Лоренца и теорема о взаимности}
  Лемма Лоренца:
  \begin{quote}
    если есть два независимых тока, создающих по отдельности соответствующие
    поля, то
    \[
      \divergence\{ \vec{E}_1\times\vec{H}_2 - \vec{E}_2\times\vec{H}_1 \} =
      \vec{j}_1^\text{ст}\cdot\vec{E}_2 - \vec{j}_2^\text{ст}\cdot\vec{E}_1.
    \]
  \end{quote}
  
  Это очевидно, т.к
  \[
    \divergence{ \vec{E}_1\times\vec{H}_2 } = \vec{H}_2\cdot\rotor\vec{E}_1 -
    \vec{E}_1\cdot\rotor\vec{H}_2.
  \]

  Теорема о взаимности:
  \[
    \int_{V_1} \vec{j}_1^\text{ст}\cdot\vec{E}_2\,dV =
    \int_{V_2} \vec{j}_2^\text{ст}\cdot\vec{E}_1\,dV.
  \]
  Эта теорема справедлива только в средах с симметричными тензорами и утверждает,
  что от перемены мест источника и приёмника качество сигнала не меняется.

\section{Задачи электродинамики}
  Ну и наконец, зачем нужна электродинамика:
  \begin{itemize}
  \item внутренняя задача: объем поля ограничен замкнутой поверхностью, вне
  которого поля не существует; решение существует и единственно если
  \begin{enumerate}
  \item если среда поглощающая или регенеративная
  \item на поверхности заданы касательные составляющие электрического или
  магнитного полей
  \end{enumerate}
  \item внешняя задача: ищется решение для неограниченного пространства при
  наличии заданных источников поля и, возможно, областей, где поле отсутствует;
  решение существует и единственно, если
    \begin{enumerate}
  \item если среда поглощающая
  \item на поверхности областей, вне которых определяется поле, заданы
  касательные составляющие электрического или магнитного полей
  \item \(\ds
        \lim_{r\to\infty}\oint_{4\pi r^2} \vec{E}\times\vec{H}^*\cdot\vec{dS}\).
  \end{enumerate}
  \end{itemize}
