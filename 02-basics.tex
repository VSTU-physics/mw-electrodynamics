\chapter{Основы}

\section{Задачи электродинамики}
  Начнём с задач, которые решает электродинамика. Задачи состоят в определении поля в некоторой области, при этом существуют 2 принципиально различных вида задач:
  \begin{itemize}
  \item внутренняя задача: объем поля ограничен замкнутой поверхностью, вне
  которого поля не существует; решение существует и единственно если:
  \begin{enumerate}
  \item если среда поглощающая или регенеративная;
  \item на поверхности заданы касательные составляющие электрического или
  магнитного полей;
  \end{enumerate}
  \item внешняя задача: ищется решение для неограниченного пространства при
  наличии заданных источников поля и, возможно, областей, где поле отсутствует;
  решение существует и единственно, если:
    \begin{enumerate}
  \item если среда поглощающая;
  \item на поверхности областей, вне которых определяется поле, заданы
  касательные составляющие электрического или магнитного полей;
  \item \(\ds
        \lim_{r\to\infty}\oint_{4\pi r^2} \vec{E}\times\vec{H}^*\cdot\vec{dS} = 0 \).
  \end{enumerate}
  \end{itemize}

\section{Основные соотношения}
  Электродинамика отталкивается от \term{уравнений Максвелла}
  \[
    \begin{cases}
      \divergence\vec{B} = 0,\\
      \divergence\vec{D} = \rho,\\
      \rotor\vec{H} = \vec{j} + \pder{\vec{D}}{t},\\
      \rotor\vec{E} = -\pder{\vec{B}}{t}.
    \end{cases}
  \]
  и \term{материальных уравнений}
  \[
      \vec{D} = \eps\vec{E},\quad
      \vec{B} = \mu\vec{H}.
  \]
    
  Из уравнений Максвелла получаются граничные условия для компонент полей на границе раздела двух сред:
  \begin{gather*}
    \vec{n}\times(\vec{H}_1 - \vec{H}_2) = \vec{I}_s,\\
    \vec{n}\times(\vec{E}_1 - \vec{E}_2) = 0,\\
    \vec{n}\cdot(\vec{D}_1 - \vec{D}_2) = \rho_s,\\
    \vec{n}\cdot(\vec{B}_1 - \vec{B}_2) = 0,
  \end{gather*}

  Также из уравнений Максвелла можно получить закон сохранения энергии:
  \[
    \pder{}{t}\left(\frac{\vec{E}\cdot\vec{D}}{2} +
    \frac{\vec{H}\cdot\vec{B}}{2}\right) =
    -\divergence(\vec{E}\times\vec{H}) - \vec{j}\cdot\vec{E},
  \]
  или в интегральном виде
  \[
    \pder{}{t}\int_V\left(\frac{\vec{E}\cdot\vec{D}}{2} +
    \frac{\vec{H}\cdot\vec{B}}{2}\right)\,dV = -\oint_S\vec{\Pi}\cdot\vec{dS}
    - \int_V\vec{j}\cdot\vec{E}\,dV.
  \]

  Вектор \( \vec{\Pi} = \vec{E}\times\vec{H} \) --- вектор Умова-Пойнтинга (или просто Пойнтинга).

\section{Гармонические колебания и комплексные амплитуды}
  В силу линейности уравнений удобно рассматривать решения уравнений в виде гармонических волн. Для них разработан удобный аппарат комплексных амплитуд:
  \[
    \vec{E}(\vec{r}, t) = \Re \left(\vec{E}(\vec{r})e^{i\omega t} \right)
  \]
  Везде далее, если не оговорено обратное, будут рассматриваться комплексные амплитуды. При таком рассмотрении \(\pder{}{t} = i\omega \) и уравнения несколько упрощаются:
  \[
    \begin{cases}
      \divergence \vec{B} = 0\\
      \divergence\vec{D} = \rho\\
      \rotor\vec{H} = \vec{j}^\text{ст} + \sigma\vec{E} + i\omega\vec{D}\\
      \rotor\vec{E} = -i\omega\vec{B}.
    \end{cases}
  \]
  \[
    \vec{D} = \eps e^{i\Delta_\eps}\vec{E},\quad
    \vec{B} = \mu e^{i\delta_\mu}\vec{H}.
  \]
  Если подставить их в уравнения Максвелла, то:
  \[
  \begin{cases}
    \divergence\mu e^{i\delta_\mu}\vec{H} = 0,\\
    \divergence\eps e^{i\Delta_\eps}\vec{E} = \rho,\\
    \rotor\vec{H} = \vec{j}^\text{ст} + i\omega\eps e^{i\Delta_\eps}\vec{E},\\
    \rotor\vec{E} = -i\omega\mu e^{i\delta_\mu}\vec{H},
  \end{cases}
  \]
  где уже
  \[
    \eps = \eps' - i\eps'', \mu = \mu' - i\mu'',
  \]
  \[
    \eps' = \eps\cos\Delta_\eps,\quad\eps'' = \eps\sin\Delta_\eps +
    \frac{\sigma}{\omega},
  \]
  \[
    \mu' = \eps\cos\delta_\mu,\quad\mu'' = \eps\sin\delta_\mu.
  \]

  Так как \( \average{\cos^2{\omega t}} = 1/2 \), то для
  средних плотностей энергии:
  \[
    \average{w_E} = \frac{1}{4}\eps'\vec{E}\vec{E}^*,\quad
    \average{w_H} = \frac{1}{4}\mu'\vec{H}\vec{H}^*,
  \]
  а комплексный вектор Пойнтинга имеет смысл интенсивности:
  \[
    \vec{\Pi} = \frac{1}{2}\vec{E}\times\vec{H}^*.
  \]

  Закон сохранения энергии
  \[
    i\omega\int_V\left(\frac{\eps\vec{E}\cdot\vec{E}^*}{2} -
    \frac{\mu^*\vec{H}\cdot\vec{H}^*}{2}\right)\,dV =
    -\oint_S\vec{\Pi}\cdot\vec{dS} -\frac{1}{2}\int_V\vec{j}\cdot\vec{E}^*\,dV.
  \]
  Выделим действительные и мнимые части и получим балансы активной и реактивной мощности:
  \[
    \omega\int_V\left(\frac{\eps''\vec{E}\cdot\vec{E}^*}{2} +
    \frac{\mu''\vec{H}\cdot\vec{H}^*}{2}\right)\,dV =
    -\Re\oint_S\vec{\Pi}\cdot\vec{dS} - \Re P_\text{вз},
  \]
  \[
    \omega\int_V\left(\frac{\eps'\vec{E}\cdot\vec{E}^*}{2} -
    \frac{\mu'\vec{H}\cdot\vec{H}^*}{2}\right)\,dV =
    -\Im\oint_S\vec{\Pi}\cdot\vec{dS} - \Im P_\text{вз}.
  \]
