\chapter{Диэлектрические волноводы}

Рассмотрим оптически более плотный волновод в среде. В силу явления полного внутреннего отражения в нём может распространяться волна.

\[
	\sin\phi_0 = n_1 / n_2.
\]

При таких переотражениях получается волна, распространяющаяся вдоль волновода.

Фазовая скорость волны равна
\[
	v_p = \frac{u_2}{\sin\phi} \le u_1,
\]
\[
	u_1 \ge v_p \ge u_2.
\]

Угол \( \phi \) падения связан с размерами и частотой волны. Если \( \phi < \phi_0 \), то волна рассеивается в окружающее пространство. Следовательно, получается отсечка.

\[
	h^2 = k_1^2 - g_1^2 = k_2^2 - g_2^2.
\]
\[
	g_2^2 > 0,\ g_1^2 < 0,
\]
\[
	g_1 = iq,\ g_2 = g.
\]

\[
	\eps_1\mu_1\omega^2 + q^2 = \eps_2\mu_2\omega^2 - g^2.
\]
Пусть \( a \) -- характерный размер волновода, тогда
\[
	\overline{g} = ag,\quad \overline{q} = aq,
\]
\[
	\overline{q}^2 + \overline{g}^2 = a^2\omega^2\eps_1\mu_1(\overline{\eps}\overline{\mu} - 1) = \overline{k}_1^2(\overline{\eps}\overline{\mu} - 1) = \overline{R}^2.
\]
\(\overline{R}\) -- нормированная частота.

Рассмотрим плоский волновод:
\[
	\Delta_\perp E_{z1} + g^2 E_{z1} = 0,
\]
\[
	\Delta_\perp E_{z2} - q^2 E_{z2} = 0,
\]
при граничном условии
\begin{align*}
	E_{z1}(x=a) = E_{z2}(x=a),\\
	E_{z1}(x=-a) = E_{z2}(x=-a),\\
	H_{y1}(x=a) = H_{y2}(x=a),\\
	H_{y1}(x=-a) = H_{y2}(x=-a).
\end{align*}

\begin{align*}
	E_{z1} = A_1\cos(gx) + B_1\sin(gx),\\
	E_{z2} = A_2 e^{-qx} + B_2 e^(qx).\\
\end{align*}

Поперечные поля имеют вид
\begin{align*}
	E_{x1} = -\frac{ih}{g}(-A_1\sin gx + B_1\cos gx),\\
	H_{y1} = -\frac{i\omega\eps_1}{g}(-A_1\sin gx + B_1\cos gx)\\
\end{align*}

Если \(A_1 = 0\), то волны называют чётными (e), \( B_1 = 0 \) -- нечётными (o).

Во второй области
\begin{align*}
	E_{x2} = \frac{ih}{q}(-A_2 e^{-qx} + B_2 e^{qx}),\\
	H_{y2} = \frac{i\omega\eps_2}{q}(-A_2 e^{-qx} + B_2 e^{qx})\\
\end{align*}

В полупространстве \( x>0 \) в силу конечности поля на бесконечности
\[
	E_{x2} = -\frac{ih}{q} A_2 e^{-qx},\quad
	H_{y2} = -\frac{i\omega\eps_2}{q} A_2 e^{-qx}
\]

Сшиваем чётную волну:
\[
	B_1\sin ga = A_2e^{-qa},\quad -\frac{\eps_1}{g} B_1\cos ga = -\frac{\eps_2}{q} A_2 e^{-qa},
\]
\[
	\tg ga = \frac{\eps_1}{\eps_2}\cdot\frac{q}{g}.
\]

Для нечётных
\[
	\ctg ga = -\frac{\eps_2}{\eps_1}\cdot\frac{q}{g}.
\]

Тут картинки с графическим решением дисперсионных уравнений.

Теперь H-волны:
\[
	H_{z1} = A_1 \cos gx + B_1 \sin gx,\quad H_{z2} = A_2 e^{-gx} + B_2 e^{gx}
\]
поперечные поля
\begin{align*}
	E_{y1} = \frac{i\omega\mu}{g}(-A_1\sin gx + B_1\cos gx),\\
	H_{x1} = \frac{-ih}{g}(-A_1\sin gx + B_1\cos gx),
\end{align*}

\begin{align*}
	E_{y2} = \frac{-i\omega\mu}{q}(-A_2 e^{-gx} + B_2 e^{gx}),\\
	H_{x2} = \frac{ih}{q}(-A_2 e^{-qx} + B_2 e^{qx}),
\end{align*}

Рассмотрим снова то же полупространство (\( B_2 = 0 \)) и граничные условия
\begin{align*}
	\mu_1H_{x1}(x=a) = \mu_2H_{x2}(x=a),\\
	H_{z1}(x=a) = H_{z2}(x=a),\\
\end{align*}

Чётные волны (\( A_1 = 0 \))
\begin{align*}
	\frac{\mu_1}{g}B_1\cos ga = -\frac{\mu_2}{q}(-A_2 e^{-qa}),\\
	B_1 \sin gx = A_2 e^{-qx},\\
\end{align*}
откуда
\[
	\frac{\mu_1}{g}\cos ga = \frac{\mu_2}{q}\sin ga
\]
\[
	\tg ga = \frac{\mu_1}{\mu_2}\cdot\frac{q}{g}
\]
Для нечётных
\[
	\ctg ga = -\frac{\mu_2}{\mu_1}\cdot\frac{q}{g}.
\]