\section{Затухание}
Причины:
\begin{itemize}
	\item потери с среде
	\item потери в металле
	\item излучение в пространство
\end{itemize}

Чтобы учесть потери можно рассмотреть
\[
	h = h' - ih''
\]

Мощность потерь
\[
	P = P_d + P_m + P_e,
\]

Поверхностный ток на стенках волновода
\[
	\vec{i}_s = \vec{n}\times\vec{H}_\tau.
\]

Для потерь вдоль волновода имеет место соотношение
\[
	\der{P}{z} = -2h''P = -\der{P_m}{z}.
\]

\[
	\der{P}{z} = -\frac{1}{2}\Re Z_{Me}\oint_C |H_\tau^2| dl,
\]

\[
	h'' = \frac{1}{2}\frac{\Re Z_{Me}\oint_C |H_\tau^2| dl}{\iint_S \Pi_z dS}.
\]

\[
	\oint_C |H_\tau^2| dl = 2 \left( \int_{y=0, x=0}^{x=a} H_x^2 dx +
	\int_{x=0, y=0}^{y=b} H_y^2 dx  \right).
\]

\[
\int_{y=0, x=0}^{x=a} H_x^2 dx = \left(\frac{\omega\eps n\pi}{g_{m,n}^2b}\right)^2\int_{y=0, x=0}^{x=a} E_0^2\sin^2\frac{m\pi x}{a} dx = \frac{a}{2}\left(\frac{\omega\eps n\pi E_0}{g_{m,n}^2b}\right)^2.
\]
\[
	\int_{x=0, y=0}^{y=b} H_y^2 dx = \frac{b}{2}\left(
	\frac{\omega\eps m\pi}{g_{m,n}^2a}E_0\right)^2,
\]
\[
	\oint_C |H_\tau^2| dl =\left(\frac{\omega\eps \pi}{g_{m,n}^2}E_0\right)^2 \frac{n^2a^3 + m^2b^3}{a^2b^2}.
\]
Знаменатель:
\[
	\iint_S (E_xH_y^* - E_yH_x^*) dS = \frac{\omega\eps}{h}
	\left(\frac{h\pi}{g_{m,n}^2}E_0\right)^2\frac{ab}{4}
	\left(\frac{m^2}{a^2}+\frac{n^2}{b^2}\right) =
	\frac{\omega\eps}{h}
	\left(\frac{h}{g_{m,n}}E_0\right)^2\frac{ab}{4},
\]
откуда
\[
	h'' = \frac{2Z_{Me}\frac{\omega\eps \pi^2}{g_{m,n}^2} (n^2a^3 + m^2b^3)}{ha^3b^3}
\]

Если ввести сквозную нумерацию гармоник, то
\[
	\int (\vec{E}_m\times\vec{H}_n^*)\cdot\vec{dS}_\perp = \delta_{mn}N_m.
\]

Рассмотрим подробнее
\[
	(\vec{E}_m\times\vec{H}_n^*)\cdot\vec{z}_0 =
	\begin{vmatrix}
		E_{mx} & E_{my}\\
		H_{nx}^* & H_{ny}^*\\
	\end{vmatrix}
	=
	E_{mx}H_{ny}^* - E_{my}H_{nx}^*.
\]

Моды являются собсвенными функциями системы уравнений
\[
	\Delta_\perp E_{m\alpha} + g_m^2 E_{m\alpha} = 0,\quad \Delta_\perp H_{n\alpha}^* + g_n^2 H_{n\alpha}^* = 0.
\]
с нужными граничными условиями.

Помножим некоторые уравнения, чтобы получить:
\[
	H_{ny}^*\Delta_\perp E_{mx} + g_m^2 H_{ny}^*E_{mx} = 0,
\]
\[
	H_{nx}^*\Delta_\perp E_{my} + g_m^2 H_{nx}^*E_{my} = 0,
\]
откуда
\[
	g_m^2(E_{mx}H_{ny}^* - E_{my}H_{nx}^*) = H_{nx}^*\Delta_\perp E_{my} - H_{ny}^*\Delta_\perp E_{mx}
\]
Аналогично
\[
	g_n^2(E_{mx}H_{ny}^* - E_{my}H_{nx}^*) = \Delta_\perp H_{nx}^* E_{my} - \Delta_\perp H_{ny}^* E_{mx}
\]

Теперь:
\begin{align*}
	&(g_m^2 - g_n^2)(E_{mx}H_{ny}^* - E_{my}H_{nx}^*) =\\
	& = \int_{S_\perp} dS \left( H_{nx}^*\Delta_\perp E_{my} - \Delta_\perp H_{nx}^* E_{my} \right)
	-
	\int_{S_\perp} dS \left( H_{ny}^*\Delta_\perp E_{mx} - \Delta_\perp H_{ny}^* E_{mx} \right) =\\
	& = \int_C dl \left(H_{nx}^* \pder{E_{my}}{n} - E_{my}\pder{H_{ny}^*}{n}\right) -
	\int_C dl \left( H_{ny}^*\pder{E_{mx}}{n} - \pder{H_{ny}^*}{n} E_{mx} \right)
\end{align*}


Рассчитаем потери в круглом волноводе (на гармонике \(E_{01}\)):
\[
	h'' = \frac{\Re Z_m}{2}\frac{\oint dl |\vec{H_\tau}|^2}{\iint \vec{dS}\cdot(\vec{E}\times\vec{H}^*)},
\]
\begin{align*}
	E_r &= -\frac{iha}{\nu_{01}}E_0 J_0'(\frac{\nu_{01}}{a}r),\\
	H_\phi &= \frac{\omega\eps}{h}E_r.
\end{align*}
\[
	\oint dl |\vec{H_\tau}|^2 = \int_0^{2\pi} a d\phi\,|H_\phi|^2 =
	\frac{a^3\omega^2\eps^2}{\nu_{01}^2}E_0^2J_0'^2(\nu_{01})\int_0^{2\pi} d\phi = \frac{2\pi a^3\omega^2\eps^2}{\nu_{01}^2}E_0^2J_0'^2(\nu_{01})
\]

\[
	\iint \vec{dS}\cdot(\vec{E}\times\vec{H}^*) = \int_0^{2\pi}d\phi\int_0^a rdr \frac{ha^2\omega\eps}{\nu_{01}^2}E_0^2J_0'^2(\frac{\nu_{01}}{a}r)= \frac{2\pi ha^2\omega\eps}{\nu_{01}^2}E_0^2 \int_0^a rdrJ_0'^2(\frac{\nu_{01}}{a}r)
\]
Рассмотрим подробнее интеграл:
\[
	\int_0^a rdr J_0'^2(\frac{\nu_{01}}{a}r) =
	\frac{a^2}{\nu_{01}^2}\int_0^{\nu_{01}} J_0'^2(x) x dx =
	\frac{a^2}{2}J_0'^2(\nu_{01}).
\]
Собирая всё вместе, получаем:
\[
	h'' = \frac{\Re Z_m}{2}\frac{\frac{2\pi a^3\omega^2\eps^2}{\nu_{01}^2}E_0^2J_0'^2(\nu_{01})}{\frac{2\pi ha^2\omega\eps}{\nu_{01}^2}E_0^2\frac{a^2}{2}J_0'^2(\nu_{01})} = \frac{\Re Z_m}{2}\frac{\omega\eps}{ha}.
\]

\begin{problem}
	Определить число мод, возбуждаемых в волноводе 72 на 34 мм на частоте 10 ГГц.
\end{problem}

\begin{problem}
	Определить структуру токов в стенках волновода.
\end{problem}

\begin{problem}
	Определить размер квадратного волновода, если на частоте 7.5 ГГц фазовая скорость волны \(E_{21}\) равна \( u = 2.5c \).
\end{problem}
\[
	g^2 = \pi^2\frac{n^2 + m^2}{a^2} = \omega^2/c^2 - h^2 = 4\pi^2f^2/c^2 \frac{5.25}{6.25},\ a^2 = \frac{6.25 c^2}{5.25}\frac{n^2 + m^2}{4f^2},
\]
\[
	a = \sqrt{\frac{6.25}{5.25}}\frac{c\sqrt{n^2 + m^2}}{2f} = 0.049~\text{м}.
\]
