\documentclass[oneside]{book}
\usepackage[russian]{babel}
\usepackage[utf8]{inputenc}
\usepackage[T2A]{fontenc}
\usepackage{indentfirst}
\usepackage{hyperref}
\usepackage[left=3cm, top=1.5cm, bottom=1.5cm, right=3cm]{geometry}
\usepackage[derivative, vectors, complex]{hedmaths}


\makeatletter
  \let\ps@plain\ps@empty
\makeatother

\usepackage{fancyhdr}
\pagestyle{fancy}
\fancyhead[LE,RO]{\thepage}
\fancyhead[LO,RE]{\leftmark}
\fancyfoot[C]{}
\renewcommand{\headrulewidth}{0.4pt}

\usepackage{amsmath}
\usepackage{amsthm}
\newtheoremstyle{problemstyle}  % <name>
        {3pt}                                               % <space above>
        {3pt}                                               % <space below>
        {\normalfont}                               % <body font>
        {}                                                  % <indent amount}
        {\bfseries}                 % <theorem head font>
        {\bfseries.}         % <punctuation after theorem head>
        {.5em}                                          % <space after theorem head>
        {}                                                  % <theorem head spec (can be left empty, meaning `normal')>
\theoremstyle{problemstyle}
\newtheorem{problem}{Задача}

\usepackage{tikz}
\usepackage{pgfplots}

\renewcommand{\vec}\boldsymbol
\newcommand{\term}\textbf

\tolerance=9999
\begin{document}
\title{Электродинамика СВЧ}
\author{В. Динамик, А. Электрик}
\maketitle
\chapter*{Несколько слов об этом}
\addcontentsline{toc}{chapter}{Несколько слов об этом}

\begin{flushright}
{\scriptsize Наука должна быть веселая, увлекательная и простая. Таковыми же должны быть и ученые.\\
П. Л. Капица}
\end{flushright}

    Привет читатель! Если ты читаешь это, то мы тебе не завидуем. Сегодня мало
    кому придёт в голову читать брошюрку двух никому неизвестных ребят,
    посвящённую электродинамике СВЧ. Чтобы помочь читателю справиться с горем,
    которое сподвигло его к чтению этой книги, мы на пальцах и с юмором
    пробежимся по основным аспектам и особенностям миллиметровых волноводов,
    потому как больше писать тут просто не о чем. Мы постарались сделать книгу
    легко читаемой: для понимания, читатель должен быть в курсе, что такое
    \emph{частная производная}, знать, что \( i^2 = -1 \) и что
    \emph{\bfseries на ноль делить нельзя}.

    Авторы этого произведения физической мысли благодарны С****** за то, что он
    настолько крут, что авторы ему за это благодарны.

    Книга написана по мотивам лекций по курсу (барабанная дробь) электродинамики
    СВЧ нашего любимого преподавателя М. В. Г******. Желаем вам приятного
    времяпреповождения!



\tableofcontents
\chapter*{Введение}
\addcontentsline{toc}{chapter}{Введение}
Для начала, стоит уточнить, что такое СВЧ и с чем его едят. Диапазон СВЧ
(сверхвысоких частот) простирается от 3 до 30 ГГц, а соответствующие длины волн
составляют от 1 до 10 см. За это его называют сантиметровым диапазоном.

Но в природе нет строгих границ диапазонов, поэтому под СВЧ в контексте этой
книги понимается более широкая часть спектра, включающая в себя
  \begin{itemize}
    \item Дециметровый диапазон: 0,3 -- 3 ГГц (УВЧ)
    \item Сантиметровый диапазон: 3 -- 30 ГГц (СВЧ)
    \item Миллиметровый диапазон: 30 -- 300 ГГц (КВЧ)
  \end{itemize}



\part{Основы электродинамики}
\chapter{Основные уравнения}
\section{Уравнения Максвелла}
    В основе электродинамики лежат уравнения Максвелла -- 3 экспериментальных
    закона (Кулона, Био-Савара-Лапласа, Фарадея) и факт отсутствия магнитных
    монополей, которые были облачены в не очень удобную математическую форму
    Максвеллом. А потом Хевисайд придумал для них не просто удобную, а краткую и
    красивую запись, в котором они дошли до нас:

    \[
      \begin{cases}
        \divergence \vec{B} = 0\\
        \divergence\vec{D} = \rho\\
        \rotor\vec{H} = \vec{j} + \pder{\vec{D}}{t}\\
        \rotor\vec{E} = -\pder{\vec{B}}{t},
      \end{cases}
    \]

    Если читатель не в курсе, что означают эти буквы, то советуем ему перестать
    это читать, собрать рюкзак и не прогуливать больше уроки!

    Там в правой части фигурирует плотность тока, которая в общем случае состоит
    из тока проводимости, конвекционного и стороннего тока:
    \[
        \vec{j} = \vec{j}^\text{пр} + \vec{j}^\text{к} + \vec{j}^\text{ст},
    \]
    \[
        \vec{j}^\text{пр} = \sigma\vec{E}, \vec{j}^\text{к} = \rho\vec{v}.
    \]

\section{Материальные уравнения}
    Индукции связаны с напряжённостями материальными уравнениями
    \[
        \vec{D} = \eps\vec{E},\quad
        \vec{B} = \mu\vec{H}.
    \]
    Проницаемости здесь абсолютные. Любителям СГС должно быть всё равно.
    Профессионалы СГС до сих пор не дочитают, так как они уже заметили,
    что уравнения Максвелла записаны неверно.

    \textit{На заметку: если Вы не совсем уверены, проводник перед Вами или нет, то есть
    простой способ проверить. Спросите, сколько у него \(\omega\eps/\sigma\).
    Если порядка 10 или больше --- то это изолятор. У проводников эта величина
    существенно меньше 1 и они обычно не афишируют это.}

    Плавно перейдём к ситуации, когда есть две контактирующие однородные среды
    и посмотрим, что там происходит с полями.

\section{Граничные условия}
    Если две среды контактируют друг с другом, то должны выполняться граничные
    условия, следующие непосредственно из уравнений Максвелла:
    \[
      \vec{n}\times(\vec{H}_1 - \vec{H}_2) = \vec{I}_s,
    \]
    \[
      \vec{n}\times(\vec{E}_1 - \vec{E}_2) = 0,
    \]
    \[
      \vec{n}\cdot(\vec{D}_1 - \vec{D}_2) = \rho_s,
    \]
    \[
      \vec{n}\cdot(\vec{B}_1 - \vec{B}_2) = 0.
    \]
    Нормаль \(\vec{n}\) торчит из границы в первую среду.

\section{Закон сохранения энергии}
    Из уравнений Максвелла следует закон сохранения энергии:
    \[
      \pder{}{t}\left(\frac{\vec{E}\cdot\vec{D}}{2} +
      \frac{\vec{H}\cdot\vec{B}}{2}\right) =
      -\divergence(\vec{E}\times\vec{H}) - \vec{j}\cdot\vec{E},
      \quad\textit{(в профиль)}
    \]
    \[
      \pder{}{t}\int_V\left(\frac{\vec{E}\cdot\vec{D}}{2} +
      \frac{\vec{H}\cdot\vec{B}}{2}\right)\,dV = -\oint_S\vec{\Pi}\cdot\vec{dS}
      - \int_V\vec{j}\cdot\vec{E}\,dV.
      \quad\textit{(в анфас)}
    \]

    Вектор \( \vec{\Pi} = \vec{E}\times\vec{H} \) --- вектор Умова-Пойнтинга,
    который обычно просто называют вектором Пойнтинга (санкции?).

\section{Гармонические колебания и комплексные амплитуды}
  Перейдём к полям, меняющимся по гармоническому закону. Их удобнее описывать на
  языке комплексных амплитуд. Если Вы внимательно читали минимальные необходимые
  требования к читателю, то Вас это не смутит такой переход:
  \[
    \vec{E}(\vec{r}, t) = \Re \left(\vec{E}(\vec{r})e^{i\omega t} \right)
  \]
  Используя этот аппарат можно заменить оператор \(\pder{}{t} = i\omega \)
  и переписать уравнения Максвелла (для среды, в которой нет конвекционного
  тока):
  \[
    \begin{cases}
      \divergence \vec{B} = 0\\
      \divergence\vec{D} = \rho\\
      \rotor\vec{H} = \vec{j}^\text{ст}+\sigma\vec{E} + i\omega\vec{D}\\
      \rotor\vec{E} = -i\omega\vec{B}.
    \end{cases}
  \]
  Материальные уравнения станут связывать комплексные амплитуды
  \[
    \vec{D} = \eps e^{i\Delta_\eps}\vec{E},\quad
    \vec{B} = \mu e^{i\delta_\mu}\vec{H}.
  \]
  Если подставить их в уравнения Максвелла, то интереснее, увы, не станет:
  \[
  \begin{cases}
    \divergence\mu\vec{H} = 0\\
    \divergence\eps\vec{E} = \rho\\
    \rotor\vec{H} = \vec{j}^\text{ст} + i\omega\eps\vec{E}\\
    \rotor\vec{E} = -i\omega\mu\vec{H}.
  \end{cases}
  \]
  Зато теперь есть повод ввести комплексные проницаемости:
  \[
    \eps = \eps' - i\eps'', \mu = \mu' - i\mu''.
  \]
  И связать их с теми странными параметрами из материальных уравнений:
  \[
    \eps' = \eps\cos\Delta_\eps,\quad\eps'' = \eps\sin\Delta_\eps +
    \frac{\sigma}{\omega},
  \]
  \[
    \mu' = \eps\cos\delta_\mu,\quad\mu'' = \eps\sin\delta_\mu.
  \]

  Поговорим теперь о средних. Так как \( \average{\cos^2{\omega t}} = 1/2 \), то
  средние плотности энергии:
  \[
    \average{w_E} = \frac{1}{4}\eps'\vec{E}\vec{E}^*,\quad
    \average{w_H} = \frac{1}{4}\mu'\vec{H}\vec{H}^*.
  \]
  Комплексный вектор Пойнтинга уже имеет смысл среднего:
  \[
    \vec{\Pi} = \frac{1}{2}\vec{E}\times\vec{H}^*.
  \]
  Учитывая это, вспомним про закон сохранения энергии:
  \[
    i\omega\int_V\left(\frac{\eps\vec{E}\cdot\vec{E}^*}{2} -
    \frac{\mu^*\vec{H}\cdot\vec{H}^*}{2}\right)\,dV =
    -\oint_S\vec{\Pi}\cdot\vec{dS} -\frac{1}{2}\int_V\vec{j}\cdot\vec{E}^*\,dV.
  \]
  Для развлечения выделим действительные и мнимые части и получим балансы
  активной и реактивной мощности:
  \[
    -\omega\int_V\left(\frac{\eps''\vec{E}\cdot\vec{E}^*}{2} +
    \frac{\mu''\vec{H}\cdot\vec{H}^*}{2}\right)\,dV =
    \Re\oint_S\vec{\Pi}\cdot\vec{dS} + \Re P_\text{вз},
  \]
  \[
    \omega\int_V\left(\frac{\eps'\vec{E}\cdot\vec{E}^*}{2} -
    \frac{\mu'\vec{H}\cdot\vec{H}^*}{2}\right)\,dV =
    -\Im\oint_S\vec{\Pi}\cdot\vec{dS} - \Im P_\text{вз}.
  \]

\section{Лемма Лоренца и теорема о взаимности}
  Лемма Лоренца:
  \begin{quote}
    если есть два независимых тока, создающих по отдельности соответствующие
    поля, то
    \[
      \divergence\{ \vec{E}_1\times\vec{H}_2 - \vec{E}_2\times\vec{H}_1 \} =
      \vec{j}_1^\text{ст}\cdot\vec{E}_2 - \vec{j}_2^\text{ст}\cdot\vec{E}_1.
    \]
  \end{quote}
  
  Это очевидно, т.к
  \[
    \divergence{ \vec{E}_1\times\vec{H}_2 } = \vec{H}_2\cdot\rotor\vec{E}_1 -
    \vec{E}_1\cdot\rotor\vec{H}_2.
  \]

  Теорема о взаимности:
  \[
    \int_{V_1} \vec{j}_1^\text{ст}\cdot\vec{E}_2\,dV =
    \int_{V_2} \vec{j}_2^\text{ст}\cdot\vec{E}_1\,dV.
  \]
  Эта теорема справедлива только в средах с симметричными тензорами и утверждает,
  что от перемены мест источника и приёмника качество сигнала не меняется.

\section{Задачи электродинамики}
  Ну и наконец, зачем нужна электродинамика:
  \begin{itemize}
  \item внутренняя задача: объем поля ограничен замкнутой поверхностью, вне
  которого поля не существует; решение существует и единственно если
  \begin{enumerate}
  \item если среда поглощающая или регенеративная
  \item на поверхности заданы касательные составляющие электрического или
  магнитного полей
  \end{enumerate}
  \item внешняя задача: ищется решение для неограниченного пространства при
  наличии заданных источников поля и, возможно, областей, где поле отсутствует;
  решение существует и единственно, если
    \begin{enumerate}
  \item если среда поглощающая
  \item на поверхности областей, вне которых определяется поле, заданы
  касательные составляющие электрического или магнитного полей
  \item \(\ds
        \lim_{r\to\infty}\oint_{4\pi r^2} \vec{E}\times\vec{H}^*\cdot\vec{dS}\).
  \end{enumerate}
  \end{itemize}

\chapter{Распространение волн}
Рассмотрим распространение плоской волны вдоль оси \(Oz\). При этом

\[
    \pder{}{x} = \pder{}{y} = 0,
\]

\[
    \begin{cases}
        -\pder{H_y}{z} = i\omega\eps E_x,\quad \pder{H_x}{z} = i\omega\eps E_y,\\
        -\pder{E_y}{z} = -i\omega\mu H_x,\quad \pder{E_x}{z} = -i\omega\mu H_y,\\
        0 = i\omega\eps E_z,\quad 0 = -i\omega\mu H_z.
    \end{cases}
\]
Отсюда
\[
    \begin{cases}
        \ppder{E_x}{z} + k^2 E_x = 0,\\
        \ppder{H_y}{z} + k^2 H_y = 0,\\
    \end{cases}
\]
где \( k = \sqrt{\eps\mu} = \omega n / c \).
Решение ищем в виде суперпозиции волн, бегущих в разных направлениях вдоль оси
\( Oz \):

\[
    E_x = Ae^{-ikz} + Be^{ikz},\quad H_y = Z_c^{-1} (Ae^{-ikz} - Be^{ikz}),
\]

где \( Z_c = \sqrt{\mu / \eps} \) -- характеристическое сопротивление среды.
Далее будем рассматривать только падающую волну:
\[
    E_x = Ae^{-ikz},\quad H_y = Z_c^{-1} Ae^{-ikz}.
\]
В непоглощающей среде \(\eps\) и \(\mu\) -- вещественные, поэтому и \(Z_c\)
также вещественно.

С учётом временной зависимости:
\[
    E_x = Ae^{-i(kz-\omega t)},\quad H_y = Z_c^{-1} Ae^{-i(kz-\omega t)}.
\]

Фазовая скорость волны определяется из всем известной формулы:
\[
v_f=\frac{dz}{dt}=\frac{\omega}{k}=\frac{c}{n_s}=u,
\]
из которой получается, что фазовая скорость численно равна скорости света в среде.

Групповая скорость волны определяется из не менее известной формулы:
\[
v_g=\frac{d\omega}{dk}=v_f - \lambda_s\frac{dv_f}{d\lambda_s},
\]

где $\lambda_s$ --- длина волны в среде.


Последнее соотношение носит гордое имя формулы Рэлея.

\part{Волноводы}
\chapter{Волны в волноводе}
\section{В прямоугольном волноводе}

\begin{gather*}
	E_x = \frac{-i}{g^2}\left(h\pder{E_z}{x} + \omega\mu\pder{H_z}{y}\right),\\
	E_y = \frac{-i}{g^2}\left(h\pder{E_z}{y} - \omega\mu\pder{H_z}{x}\right),\\
	H_x = \frac{i}{g^2}\left(\omega\eps\pder{E_z}{y} - h\pder{H_z}{x}\right),\\
	H_y = \frac{-i}{g^2}\left(\omega\eps\pder{E_z}{x} + h\pder{H_z}{y}\right).
\end{gather*}

\section{В круглом волноводе}
\begin{gather*}
	E_r =    \frac{-i}{g^2}\left(h\pder{E_z}{r} + \frac{\omega\mu}{r}\pder{H_z}{\phi}\right),\\
	E_\phi = \frac{-i}{g^2}\left(\frac{h}{r}\pder{E_z}{\phi} - \omega\mu\pder{H_z}{r}\right),\\
	H_r =    \frac{i}{g^2}\left(\frac{\omega\eps}{r}\pder{E_z}{\phi} - h\pder{H_z}{r}\right),\\
	H_\phi = \frac{-i}{g^2}\left(\omega\eps\pder{E_z}{r} + \frac{h}{r}\pder{H_z}{\phi}\right).
\end{gather*}

Из уравнения Гельмгольца получаем для продольных компонент поля
\begin{gather}
	\Delta_\perp E_z + g^2 E_z = 0,\\
	\Delta_\perp H_z + g^2 H_z = 0.
\end{gather}

В системе могут существовать отдельные TE и TM волны, если на контуре волновода
\[
	\pder{E_z}{l} = \pder{H_z}{l} = 0.
\]
Если условие не выполняются, то в системе будут наблюдаться гибридные волны.

\section{Прямоугольном волновод}

Рассмотрим прямоугольный волновод \( a \times b, a > b, a || Ox \).

E-волны (TM) имеют в нём следующий вид:

\begin{align*}
	& E_z = E_0\sin\frac{m\pi x}{a}\sin\frac{n\pi y}{b},\\
	& H_z = 0,\\
	& E_x = -i\frac{hm\pi}{g_{m,n}^2a}E_0\cos\frac{m\pi x}{a}\sin\frac{n\pi y}{b},\\
	& E_y = -i\frac{hn\pi}{g_{m,n}^2b}E_0\sin\frac{m\pi x}{a}\cos\frac{n\pi y}{b},\\
	& H_x = -\frac{\omega\eps}{h}E_y = i\frac{\omega\eps n\pi}{g_{m,n}^2b}E_0
	 								\sin\frac{m\pi x}{a}\cos\frac{n\pi y}{b},\\
	& H_y = \frac{\omega\eps}{h}E_x = -i\frac{\omega\eps m\pi}{g_{m,n}^2a}E_0
	 								\cos\frac{m\pi x}{a}\sin\frac{n\pi y}{b}.
\end{align*}

H-волны (TE):
\begin{align*}
	& E_z = 0,\\
	& H_z = H_0\cos\frac{m\pi x}{a}\cos\frac{n\pi y}{b},\\
	& E_x = i\frac{\omega\mu n\pi}{g_{m,n}^2b}H_0\cos\frac{m\pi x}{a}\sin\frac{n\pi y}{b},\\
	& E_y = -i\frac{\omega\mu m\pi}{g_{m,n}^2a}H_0\sin\frac{m\pi x}{a}\cos\frac{n\pi y}{b},\\
	& H_x = -\frac{h}{\omega\mu}E_y = i\frac{hm\pi}{g_{m,n}^2a}H_0
	 								\sin\frac{m\pi x}{a}\cos\frac{n\pi y}{b},\\
	& H_y = \frac{h}{\omega\mu}E_x = -i\frac{h n\pi}{g_{m,n}^2b}H_0
	 								\cos\frac{m\pi x}{a}\sin\frac{n\pi y}{b}.
\end{align*}

Здесь \( g_{m,n} \) -- поперечное волновое число, определяемое выражением
\[
	g_{m,n} = \pi\sqrt{\frac{m^2}{a^2} + \frac{n^2}{b^2}}.
\]

Так как
\[
	h^2 = \beta^2 - g_{m,n}^2,
\]
то для данного типа волны существует критическая частота, ниже которой эта волна возбуждаться не может. Она называется критической:
\[
	\omega_c = \frac{c}{\sqrt{\eps_r \mu_r}}g_{m,n}.
\]

Дисперсионное соотношение имеет вид:
\[
	h^2 = \frac{\omega^2 \eps_r \mu_r}{c^2} - g_{m,n}^2,
\]
откуда
\[
	v_p = \frac{\omega}{h} = \frac{c}{\sqrt{\eps_r\mu_r}}\frac{1}{\sqrt{1 - \omega_c^2 / \omega^2}},
\]

\[
	v_g = \der{\omega}{h} = \frac{c}{\sqrt{\eps_r\mu_r}}\sqrt{1 - \omega_c^2 / \omega^2}.
\]

Заметим, что
\[
	v_p v_g = \frac{c^2}{\eps_r\mu_r} \text{ --- квадрат скорости света в среде.}
\]

\begin{center}
\begin{tikzpicture}
\begin{axis}[
	xmin = 0, xmax = 1, ymin = 0, ymax = 4,
    xlabel = {$\lambda_0 / \lambda_c$},
    ylabel = {$v / c$},
    minor tick num = 2
]
\addplot[domain=0:0.97,samples=100]{(1 - x^2)^-0.5} node[pos=0.75,pin=170:{$v_p$}]{};
\addplot[domain=0:1,samples=100]{(1 - x^2)^0.5} node[pos=0.8,pin=100:{$v_g$}]{};
\end{axis}
\end{tikzpicture}
\end{center}
\chapter{Металлические волноводы}
\section{Прямоугольный волновод}

\begin{gather*}
	E_x = \frac{-i}{g^2}\left(h\pder{E_z}{x} + \omega\mu\pder{H_z}{y}\right),\\
	E_y = \frac{-i}{g^2}\left(h\pder{E_z}{y} - \omega\mu\pder{H_z}{x}\right),\\
	H_x = \frac{i}{g^2}\left(\omega\eps\pder{E_z}{y} - h\pder{H_z}{x}\right),\\
	H_y = \frac{-i}{g^2}\left(\omega\eps\pder{E_z}{x} + h\pder{H_z}{y}\right).
\end{gather*}

В системе могут существовать отдельные TE и TM волны, если на контуре волновода
\[
	\pder{E_z}{l} = \pder{H_z}{l} = 0.
\]
Если условие не выполняются, то в системе будут наблюдаться гибридные волны.

Рассмотрим прямоугольный волновод \( a \times b, a > b, a || Ox \), представленный на рисунке ниже:

\begin{center}
\begin{tikzpicture}[scale=0.15]
	\draw[very thick] (0, 0) rectangle (23, 10);
	\draw[->] (-4, -4) -- (25, -4);
	\draw[->] (-4, -4) -- (-4, 13);
	\node[below] at (11.5, 0) {$a$};
	\node[left] at (0, 5) {$b$};
	\node[below] at (25,-4) {$x$};
	\node[left] at (-4, 13) {$y$};
\end{tikzpicture}
\end{center}

E-волны (TM) имеют в нём следующий вид:

\begin{align*}
	& E_z = E_0\sin\frac{m\pi x}{a}\sin\frac{n\pi y}{b},\\
	& H_z = 0,\\
	& E_x = -i\frac{hm\pi}{g_{m,n}^2a}E_0\cos\frac{m\pi x}{a}\sin\frac{n\pi y}{b},\\
	& E_y = -i\frac{hn\pi}{g_{m,n}^2b}E_0\sin\frac{m\pi x}{a}\cos\frac{n\pi y}{b},\\
	& H_x = -\frac{\omega\eps}{h}E_y = i\frac{\omega\eps n\pi}{g_{m,n}^2b}E_0
	 								\sin\frac{m\pi x}{a}\cos\frac{n\pi y}{b},\\
	& H_y = \frac{\omega\eps}{h}E_x = -i\frac{\omega\eps m\pi}{g_{m,n}^2a}E_0
	 								\cos\frac{m\pi x}{a}\sin\frac{n\pi y}{b}.
\end{align*}

H-волны (TE):
\begin{align*}
	& E_z = 0,\\
	& H_z = H_0\cos\frac{m\pi x}{a}\cos\frac{n\pi y}{b},\\
	& E_x = i\frac{\omega\mu n\pi}{g_{m,n}^2b}H_0\cos\frac{m\pi x}{a}\sin\frac{n\pi y}{b},\\
	& E_y = -i\frac{\omega\mu m\pi}{g_{m,n}^2a}H_0\sin\frac{m\pi x}{a}\cos\frac{n\pi y}{b},\\
	& H_x = -\frac{h}{\omega\mu}E_y = i\frac{hm\pi}{g_{m,n}^2a}H_0
	 								\sin\frac{m\pi x}{a}\cos\frac{n\pi y}{b},\\
	& H_y = \frac{h}{\omega\mu}E_x = i\frac{h n\pi}{g_{m,n}^2b}H_0
	 								\cos\frac{m\pi x}{a}\sin\frac{n\pi y}{b}.
\end{align*}

Здесь \( g_{m,n} \) -- поперечное волновое число, определяемое выражением
\[
	g_{m,n} = \pi\sqrt{\frac{m^2}{a^2} + \frac{n^2}{b^2}}.
\]

Так как
\[
	h^2 = \beta^2 - g_{m,n}^2,
\]
то для данного типа волны существует критическая частота, ниже которой эта волна возбуждаться не может. Она называется критической:
\[
	\omega_c = \frac{c}{\sqrt{\eps_r \mu_r}}g_{m,n}.
\]

Дисперсионное соотношение имеет вид:
\[
	h^2 = \frac{\omega^2 \eps_r \mu_r}{c^2} - g_{m,n}^2,
\]
откуда
\[
	v_p = \frac{\omega}{h} = \frac{c}{\sqrt{\eps_r\mu_r}}\frac{1}{\sqrt{1 - \omega_c^2 / \omega^2}},
\]

\[
	v_g = \der{\omega}{h} = \frac{c}{\sqrt{\eps_r\mu_r}}\sqrt{1 - \omega_c^2 / \omega^2}.
\]

Заметим, что
\[
	v_p v_g = \frac{c^2}{\eps_r\mu_r} \text{ --- квадрат скорости света в среде.}
\]

\begin{center}
\begin{tikzpicture}
\begin{axis}[
	xmin = 0, xmax = 1, ymin = 0, ymax = 4,
    xlabel = {$\lambda_0 / \lambda_c$},
    ylabel = {$v / c$},
    minor tick num = 2
]
\addplot[domain=0:0.97,samples=100,line width=1.5pt]{(1 - x^2)^-0.5} node[pos=0.75,pin=170:{$v_p$}]{};
\addplot[domain=0:1,samples=100,line width=1.5pt]{sqrt(1 - x^2)} node[pos=0.8,pin=100:{$v_g$}]{};
\end{axis}
\end{tikzpicture}
\end{center}


Относительное расположение критических длин волн:
\begin{center}
	\begin{tikzpicture}[>=stealth]
		\draw[->](0,0) -- (5,0);
		\draw[fill] (1,0) circle(1.5pt) node[below]{$E_{11}$};
		\draw[fill] (3,0) circle(1.5pt) node[below]{$H_{01}$};
		\draw[fill] (4,0) circle(1.5pt) node[below]{$H_{10}$};
		\draw (5,0) node[below]{$\lambda$};
	\end{tikzpicture}
\end{center}

Характеристические сопротивления для волн:
\begin{align*}
	Z_c^E = & \sqrt{\frac{E_xE_x^* + E_yE_y^*}{H_xH_x^* + H_yH_y^*}} = \frac{h}{\omega\eps} = Z_c\sqrt{1 - \frac{\lambda_0^2}{\lambda_c^2}},\\
	Z_c^H = & \sqrt{\frac{E_xE_x^* + E_yE_y^*}{H_xH_x^* + H_yH_y^*}} = \frac{\omega\mu}{h} = \frac{Z_c}{\sqrt{1 - \frac{\lambda_0^2}{\lambda_c^2}}}.
\end{align*}

Рассмотрим передачу мощности в волне \( H_{10} \):
\[
	\Pi_z = -\frac{1}{2}E_y H_x^* = \frac{1}{2}\frac{h}{\omega\mu}E_yE_y^* = \frac{h}{2\omega\mu}\omega^2\mu^2 H_0^2\sin^2\frac{\pi x}{a} = \frac{\omega\mu h}{2}H_0^2\sin^2\frac{\pi x}{a},
\]
\[
	h= \frac{2\pi}{\lambda_0}\sqrt{1 - \frac{\lambda_0^2}{4a^2}},
\]
\[
	\Pi_z = \frac{\mu ca^2}{\pi\lambda_0^2}\sqrt{1 - \frac{\lambda_0^2}{4a^2}}H_0^2\sin^2\frac{\pi x}{a},
\]
а передаваемая мощность определяется интегралом по поперечному сечению:
\[
	P = \frac{\mu ca^3b}{2\pi\lambda_0^2}\sqrt{1 - \frac{\lambda_0^2}{4a^2}}H_0^2 = \frac{abh}{4\omega\mu}E_{max}^2.
\]

\section{Коаксиальный волновод}
\begin{gather*}
	E_r =    \frac{-i}{g^2}\left(h\pder{E_z}{r} + \frac{\omega\mu}{r}\pder{H_z}{\phi}\right),\\
	E_\phi = \frac{-i}{g^2}\left(\frac{h}{r}\pder{E_z}{\phi} - \omega\mu\pder{H_z}{r}\right),\\
	H_r =    \frac{i}{g^2}\left(\frac{\omega\eps}{r}\pder{E_z}{\phi} - h\pder{H_z}{r}\right),\\
	H_\phi = \frac{-i}{g^2}\left(\omega\eps\pder{E_z}{r} + \frac{h}{r}\pder{H_z}{\phi}\right).
\end{gather*}

Основной режим работы -- TEM-волны, волноводные гармоники -- паразитные.
Для определения продольных компонент полей придётся решать уравнение Гельмгольца
\[
	\Delta_\perp E_z = g^2 E_z = 0,\quad \Delta_\perp H_z = g^2 H_z = 0.
\]
в области, огрпниченной двумя цилиндическими поверхностями
\begin{center}
\begin{tikzpicture}[scale=0.2]
	\draw[very thick] (0, 0) circle (10);
	\draw[very thick] (0, 0) circle (4);
	\draw (-12, 0) -- (12, 0);
	\draw[->] (0, 0) -- (-3.87, 1);
	\node[above] at (-1.94, 0.5) {$b$};
	\draw[->] (0, 0) -- (7.9, -5.9);
	\node[above] at (5.3, -4) {$a$};
	\draw[->] (0, 0) -- (5, 5) node[above]{$r$};
	\draw[->] (6,0) arc (0:45:6 and 6);
	\node[right] at (6, 2.33) {$\phi$};
\end{tikzpicture}
\end{center}

с граничными
\[
	\left.E_\tau\right|_{r=a,b} = 0 \Rightarrow \left.E_z\right|_{r=a,b} = 0, \left.\pder{H_z}{r}\right|_{r=a,b} = 0
\]
\[
	\frac{1}{r}\pder{}{r}\left( rE_z \right) + \frac{1}{r^2}\ppder{E_z}{\phi} + g^2E_z = 0.
\]
\[
	E_z = [AJ_m(gr) + BN_m(gr)]\Phi(m\phi)
\]
При \( b = 0 \) из условия конечности поля \( B = 0 \) и
\[
	E_z = E_0J_m(gr)\Phi(m\phi).
\]
Поперечные волновые числа определяются из условия
\[
	J_m(ga) = 0,\ g_{mn} = \frac{\nu_{mn}}{a}.
\]

\begin{align*}
	E_z &= E_0 J_m(\frac{\nu_{mn}}{a}r)\Phi(m\phi),\\
	E_r &= -\frac{iha}{\nu_{mn}}E_0 J_m'(\frac{\nu_{mn}}{a}r)\Phi(m\phi),\\
	E_\phi &= -\frac{iha^2m}{\nu_{mn}^2r}E_0 J_m(\frac{\nu_{mn}}{a}r)\Phi'(m\phi),\\
	H_z &= 0,\\
	H_r &= -\frac{\omega\eps}{h}E_\phi,\\
	H_\phi &= \frac{\omega\eps}{h}E_r.
\end{align*}


Рассмотрим теперь \( H \)-волны:
\[
	\Delta_\perp H_z + g^2H_z = 0,\quad \left.\pder{H_z}{n}\right|_{r=a,b}= 0
\]

\[
	H_z = [AJ_m(gr) + BN_m(gr)]\Phi(m\phi)
\]

Если \( b=0 \), то в силу ограниченности поля \(B = 0\), а дисперсионное соотношение имеет вид
\[
	J_m'(ga) = 0 \Rightarrow ga = \mu_{mn}, g_{mn} = \frac{\mu_{mn}}{a}, \lambda_c = \frac{2\pi a}{\mu_{mn}}, \mu_{11} = 1.841.
\]

Поля в волноводе имеют вид:
\begin{align*}
	H_z &= H_0 J_m(\frac{\mu_{mn}}{a}r)\Phi(m\phi),\\
	E_r &= -\frac{i\omega\mu a^2m}{r\mu_{mn}^2}H_0 J_m(\frac{\mu_{mn}}{a}r)\Phi'(m\phi),\\
	E_\phi &= \frac{i\omega\mu a}{\mu_{mn}}E_0 J_m'(\frac{\mu_{mn}}{a}r)\Phi(m\phi),\\
	E_z &= 0,\\
	H_r &= -\frac{h}{\omega\mu}E_\phi,\\
	H_\phi &= \frac{h}{\omega\mu}E_r.
\end{align*}

Для коаксиального волновода имеем
\[
	\begin{cases}
		AJ_m'(gb) + BN_m'(gb) = 0,\\
		AJ_m'(ga) + BN_m'(ga) = 0.\\
	\end{cases},
	\quad
	J_m'(gb)N_m'(ga) - J_m'(ga)N_m'(gb) = 0
\]

Характеристическое сопротивление волновода
\[
	Z_c = \frac{E_\perp}{H_\perp}.
\]

\section{Затухание}
Причины:
\begin{itemize}
	\item потери с среде
	\item потери в металле
	\item излучение в пространство
\end{itemize}

Чтобы учесть потери можно рассмотреть
\[
	h = h' - ih''
\]

Мощность потерь
\[
	P = P_d + P_m + P_e,
\]

Поверхностный ток на стенках волновода
\[
	\vec{i}_s = \vec{n}\times\vec{H}_\tau.
\]

Для потерь вдоль волновода имеет место соотношение
\[
	\der{P}{z} = -2h''P = -\der{P_m}{z}.
\]

\[
	\der{P}{z} = -\frac{1}{2}\Re Z_{Me}\oint_C |H_\tau^2| dl,
\]

\[
	h'' = \frac{1}{2}\frac{\Re Z_{Me}\oint_C |H_\tau^2| dl}{\iint_S \Pi_z dS}.
\]

\[
	\oint_C |H_\tau^2| dl = 2 \left( \int_{y=0, x=0}^{x=a} H_x^2 dx +
	\int_{x=0, y=0}^{y=b} H_y^2 dx  \right).
\]

\[
\int_{y=0, x=0}^{x=a} H_x^2 dx = \left(\frac{\omega\eps n\pi}{g_{m,n}^2b}\right)^2\int_{y=0, x=0}^{x=a} E_0^2\sin^2\frac{m\pi x}{a} dx = \frac{a}{2}\left(\frac{\omega\eps n\pi E_0}{g_{m,n}^2b}\right)^2.
\]
\[
	\int_{x=0, y=0}^{y=b} H_y^2 dx = \frac{b}{2}\left(
	\frac{\omega\eps m\pi}{g_{m,n}^2a}E_0\right)^2,
\]
\[
	\oint_C |H_\tau^2| dl =\left(\frac{\omega\eps \pi}{g_{m,n}^2}E_0\right)^2 \frac{n^2a^3 + m^2b^3}{a^2b^2}.
\]
Знаменатель:
\[
	\iint_S (E_xH_y^* - E_yH_x^*) dS = \frac{\omega\eps}{h}
	\left(\frac{h\pi}{g_{m,n}^2}E_0\right)^2\frac{ab}{4}
	\left(\frac{m^2}{a^2}+\frac{n^2}{b^2}\right) =
	\frac{\omega\eps}{h}
	\left(\frac{h}{g_{m,n}}E_0\right)^2\frac{ab}{4},
\]
откуда
\[
	h'' = \frac{2Z_{Me}\frac{\omega\eps \pi^2}{g_{m,n}^2} (n^2a^3 + m^2b^3)}{ha^3b^3}
\]

Если ввести сквозную нумерацию гармоник, то
\[
	\int (\vec{E}_m\times\vec{H}_n^*)\cdot\vec{dS}_\perp = \delta_{mn}N_m.
\]

Рассмотрим подробнее
\[
	(\vec{E}_m\times\vec{H}_n^*)\cdot\vec{z}_0 =
	\begin{vmatrix}
		E_{mx} & E_{my}\\
		H_{nx}^* & H_{ny}^*\\
	\end{vmatrix}
	=
	E_{mx}H_{ny}^* - E_{my}H_{nx}^*.
\]

Моды являются собсвенными функциями системы уравнений
\[
	\Delta_\perp E_{m\alpha} + g_m^2 E_{m\alpha} = 0,\quad \Delta_\perp H_{n\alpha}^* + g_n^2 H_{n\alpha}^* = 0.
\]
с нужными граничными условиями.

Помножим некоторые уравнения, чтобы получить:
\[
	H_{ny}^*\Delta_\perp E_{mx} + g_m^2 H_{ny}^*E_{mx} = 0,
\]
\[
	H_{nx}^*\Delta_\perp E_{my} + g_m^2 H_{nx}^*E_{my} = 0,
\]
откуда
\[
	g_m^2(E_{mx}H_{ny}^* - E_{my}H_{nx}^*) = H_{nx}^*\Delta_\perp E_{my} - H_{ny}^*\Delta_\perp E_{mx}
\]
Аналогично
\[
	g_n^2(E_{mx}H_{ny}^* - E_{my}H_{nx}^*) = \Delta_\perp H_{nx}^* E_{my} - \Delta_\perp H_{ny}^* E_{mx}
\]

Теперь:
\begin{align*}
	&(g_m^2 - g_n^2)(E_{mx}H_{ny}^* - E_{my}H_{nx}^*) =\\
	& = \int_{S_\perp} dS \left( H_{nx}^*\Delta_\perp E_{my} - \Delta_\perp H_{nx}^* E_{my} \right)
	-
	\int_{S_\perp} dS \left( H_{ny}^*\Delta_\perp E_{mx} - \Delta_\perp H_{ny}^* E_{mx} \right) =\\
	& = \int_C dl \left(H_{nx}^* \pder{E_{my}}{n} - E_{my}\pder{H_{ny}^*}{n}\right) -
	\int_C dl \left( H_{ny}^*\pder{E_{mx}}{n} - \pder{H_{ny}^*}{n} E_{mx} \right)
\end{align*}


Рассчитаем потери в круглом волноводе (на гармонике \(E_{01}\)):
\[
	h'' = \frac{\Re Z_m}{2}\frac{\oint dl |\vec{H_\tau}|^2}{\iint \vec{dS}\cdot(\vec{E}\times\vec{H}^*)},
\]
\begin{align*}
	E_r &= -\frac{iha}{\nu_{01}}E_0 J_0'(\frac{\nu_{01}}{a}r),\\
	H_\phi &= \frac{\omega\eps}{h}E_r.
\end{align*}
\[
	\oint dl |\vec{H_\tau}|^2 = \int_0^{2\pi} a d\phi\,|H_\phi|^2 =
	\frac{a^3\omega^2\eps^2}{\nu_{01}^2}E_0^2J_0'^2(\nu_{01})\int_0^{2\pi} d\phi = \frac{2\pi a^3\omega^2\eps^2}{\nu_{01}^2}E_0^2J_0'^2(\nu_{01})
\]

\[
	\iint \vec{dS}\cdot(\vec{E}\times\vec{H}^*) = \int_0^{2\pi}d\phi\int_0^a rdr \frac{ha^2\omega\eps}{\nu_{01}^2}E_0^2J_0'^2(\frac{\nu_{01}}{a}r)= \frac{2\pi ha^2\omega\eps}{\nu_{01}^2}E_0^2 \int_0^a rdrJ_0'^2(\frac{\nu_{01}}{a}r)
\]
Рассмотрим подробнее интеграл:
\[
	\int_0^a rdr J_0'^2(\frac{\nu_{01}}{a}r) =
	\frac{a^2}{\nu_{01}^2}\int_0^{\nu_{01}} J_0'^2(x) x dx =
	\frac{a^2}{2}J_0'^2(\nu_{01}).
\]
Собирая всё вместе, получаем:
\[
	h'' = \frac{\Re Z_m}{2}\frac{\frac{2\pi a^3\omega^2\eps^2}{\nu_{01}^2}E_0^2J_0'^2(\nu_{01})}{\frac{2\pi ha^2\omega\eps}{\nu_{01}^2}E_0^2\frac{a^2}{2}J_0'^2(\nu_{01})} = \frac{\Re Z_m}{2}\frac{\omega\eps}{ha}.
\]

\begin{problem}
	Определить число мод, возбуждаемых в волноводе 72 на 34 мм на частоте 10 ГГц.
\end{problem}

\begin{problem}
	Определить структуру токов в стенках волновода.
\end{problem}

\begin{problem}
	Определить размер квадратного волновода, если на частоте 7.5 ГГц фазовая скорость волны \(E_{21}\) равна \( u = 2.5c \).
\end{problem}
\[
	g^2 = \pi^2\frac{n^2 + m^2}{a^2} = \omega^2/c^2 - h^2 = 4\pi^2f^2/c^2 \frac{5.25}{6.25},\ a^2 = \frac{6.25 c^2}{5.25}\frac{n^2 + m^2}{4f^2},
\]
\[
	a = \sqrt{\frac{6.25}{5.25}}\frac{c\sqrt{n^2 + m^2}}{2f} = 0.049~\text{м}.
\]

\section{Волноводы странной формы}

П-образные волноводы позволяют получать разреженный спектр. Расплатой за это является меньшая мощность и большие потери.

Для расчёта такого рода волноводов приходится использовать численные методы.

\chapter{Диэлектрические волноводы}

Рассмотрим оптически более плотный волновод в среде. В силу явления полного внутреннего отражения в нём может распространяться волна.

\[
	\sin\phi_0 = n_1 / n_2.
\]

При таких переотражениях получается волна, распространяющаяся вдоль волновода.

Фазовая скорость волны равна
\[
	v_p = \frac{u_2}{\sin\phi} \le u_1,
\]
\[
	u_1 \ge v_p \ge u_2.
\]

Угол \( \phi \) падения связан с размерами и частотой волны. Если \( \phi < \phi_0 \), то волна рассеивается в окружающее пространство. Следовательно, получается отсечка.

\[
	h^2 = k_1^2 - g_1^2 = k_2^2 - g_2^2.
\]
\[
	g_2^2 > 0,\ g_1^2 < 0,
\]
\[
	g_1 = iq,\ g_2 = g.
\]

\[
	\eps_1\mu_1\omega^2 + q^2 = \eps_2\mu_2\omega^2 - g^2.
\]
Пусть \( a \) -- характерный размер волновода, тогда
\[
	\overline{g} = ag,\quad \overline{q} = aq,
\]
\[
	\overline{q}^2 + \overline{g}^2 = a^2\omega^2\eps_1\mu_1(\overline{\eps}\overline{\mu} - 1) = \overline{k}_1^2(\overline{\eps}\overline{\mu} - 1) = \overline{R}^2.
\]
\(\overline{R}\) -- нормированная частота.

Рассмотрим плоский волновод:
\[
	\Delta_\perp E_{z1} + g^2 E_{z1} = 0,
\]
\[
	\Delta_\perp E_{z2} - q^2 E_{z2} = 0,
\]
при граничном условии
\begin{align*}
	E_{z1}(x=a) = E_{z2}(x=a),\\
	E_{z1}(x=-a) = E_{z2}(x=-a),\\
	H_{y1}(x=a) = H_{y2}(x=a),\\
	H_{y1}(x=-a) = H_{y2}(x=-a).
\end{align*}

\begin{align*}
	E_{z1} = A_1\cos(gx) + B_1\sin(gx),\\
	E_{z2} = A_2 e^{-qx} + B_2 e^(qx).\\
\end{align*}

Поперечные поля имеют вид
\begin{align*}
	E_{x1} = -\frac{ih}{g}(-A_1\sin gx + B_1\cos gx),\\
	H_{y1} = -\frac{i\omega\eps_1}{g}(-A_1\sin gx + B_1\cos gx)\\
\end{align*}

Если \(A_1 = 0\), то волны называют чётными (e), \( B_1 = 0 \) -- нечётными (o).

Во второй области
\begin{align*}
	E_{x2} = \frac{ih}{q}(-A_2 e^{-qx} + B_2 e^{qx}),\\
	H_{y2} = \frac{i\omega\eps_2}{q}(-A_2 e^{-qx} + B_2 e^{qx})\\
\end{align*}

В полупространстве \( x>0 \) в силу конечности поля на бесконечности
\[
	E_{x2} = -\frac{ih}{q} A_2 e^{-qx},\quad
	H_{y2} = -\frac{i\omega\eps_2}{q} A_2 e^{-qx}
\]

Сшиваем чётную волну:
\[
	B_1\sin ga = A_2e^{-qa},\quad -\frac{\eps_1}{g} B_1\cos ga = -\frac{\eps_2}{q} A_2 e^{-qa},
\]
\[
	\tg ga = \frac{\eps_1}{\eps_2}\cdot\frac{q}{g}.
\]

Для нечётных
\[
	\ctg ga = -\frac{\eps_2}{\eps_1}\cdot\frac{q}{g}.
\]

Тут картинки с графическим решением дисперсионных уравнений.

Теперь H-волны:
\[
	H_{z1} = A_1 \cos gx + B_1 \sin gx,\quad H_{z2} = A_2 e^{-gx} + B_2 e^{gx}
\]
поперечные поля
\begin{align*}
	E_{y1} = \frac{i\omega\mu}{g}(-A_1\sin gx + B_1\cos gx),\\
	H_{x1} = \frac{-ih}{g}(-A_1\sin gx + B_1\cos gx),
\end{align*}

\begin{align*}
	E_{y2} = \frac{-i\omega\mu}{q}(-A_2 e^{-gx} + B_2 e^{gx}),\\
	H_{x2} = \frac{ih}{q}(-A_2 e^{-qx} + B_2 e^{qx}),
\end{align*}

Рассмотрим снова то же полупространство (\( B_2 = 0 \)) и граничные условия
\begin{align*}
	\mu_1H_{x1}(x=a) = \mu_2H_{x2}(x=a),\\
	H_{z1}(x=a) = H_{z2}(x=a),\\
\end{align*}

Чётные волны (\( A_1 = 0 \))
\begin{align*}
	\frac{\mu_1}{g}B_1\cos ga = -\frac{\mu_2}{q}(-A_2 e^{-qa}),\\
	B_1 \sin gx = A_2 e^{-qx},\\
\end{align*}
откуда
\[
	\frac{\mu_1}{g}\cos ga = \frac{\mu_2}{q}\sin ga
\]
\[
	\tg ga = \frac{\mu_1}{\mu_2}\cdot\frac{q}{g}
\]
Для нечётных
\[
	\ctg ga = -\frac{\mu_2}{\mu_1}\cdot\frac{q}{g}.
\]
\part{Резонаторы}
\chapter{Объёмные резонаторы}

Объёмным резонатором называют совокупность металлических и/или диэлектрических тел, внутри или вблизи которых концентрируется электромагнитное поле. Область существования поля пожно отделить от остального пространства поверхностью S, излучение энергии через которую отсутствует или незначительно.

Количество энергии, запасенной в поле резонатора зависит от частоты и вблизи некоторых частот, называемых собственными, количество запасенной энергии резко увеличивается (при прочих равных внешних условиях).

Назначение резонаторов -- запасение энергии.

По устройству резонаторы делят на открытые и закрытые. Закрытые резонаторы окружены металлическими поверхностями (кольцевые, волноводные). Открытые -- диэлектрические (волноводные, зеркальные).


\section{Колебания в резонаторах}
\[
	\Delta\vec{E} + k^2\vec{E} = 0,\quad
	\Delta\vec{H} + k^2\vec{H} = 0.
\]

Для волноводных резонаторов
\[
	\Delta E_z + k^2 E_z = 0,\quad
	\Delta H_z + k^2 H_z = 0,
\]
с граничными условиями
\[
	\left.\vec{E}_\tau\right|_S = 0.
\]
На боковых гранях
\[
	E_z = 0,\quad \pder{H_z}{n} = 0,
\]
на торцах
\[
	H_z = 0,\quad \pder{E_z}{z} = 0.
\]

Для прямоугольного резонатора:
\[
	k^2 = \pi^2\left[\left(\frac{m}{a}\right)^2+
					 \left(\frac{n}{b}\right)^2+
					 \left(\frac{p}{l}\right)^2\right],
	\omega = \frac{k}{\sqrt{\eps\mu}},
\]


Для E-колебаний
\begin{align*}
	E_z & = E_0\sin\frac{m\pi}{a}\sin\frac{n\pi}{b}\cos\frac{p\pi}{l},\\
	E_x & = -\frac{mp\pi^2}{al g^2}E_0\cos\frac{m\pi}{a}\sin\frac{n\pi}{b}\sin\frac{p\pi}{l},\\
	E_y & = -\frac{np\pi^2}{bl g^2}E_0\sin\frac{m\pi}{a}\cos\frac{n\pi}{b}\sin\frac{p\pi}{l},\\
	H_x & = \frac{i\omega\eps}{g^2}\frac{n\pi}{b}\sin\frac{m\pi}{a}\cos\frac{n\pi}{b}\cos\frac{p\pi}{l},\\
	H_y & = -\frac{i\omega\eps}{g^2}\frac{m\pi}{a}\cos\frac{m\pi}{a}\sin\frac{n\pi}{b}\cos\frac{p\pi}{l}.
\end{align*}
Низший тип таких колебаний -- \(E_{110}\).

Для H-колебаний
\begin{align*}
	E_x & = -\frac{in\pi\omega\mu}{b g^2}H_0\cos\frac{m\pi}{a}\sin\frac{n\pi}{b}\sin\frac{p\pi}{l},\\
	E_y & = -\frac{im\pi\omega\mu}{a g^2}H_0\sin\frac{m\pi}{a}\cos\frac{n\pi}{b}\sin\frac{p\pi}{l},\\
	H_z & = H_0\cos\frac{m\pi}{a}\cos\frac{n\pi}{b}\sin\frac{p\pi}{l},\\
	H_x & = -\frac{pm\pi^2}{g^2 al}H_0\sin\frac{m\pi}{a}\cos\frac{n\pi}{b}\cos\frac{p\pi}{l},\\
	H_y & = -\frac{pn\pi^2}{g^2 bl}H_0\cos\frac{m\pi}{a}\sin\frac{n\pi}{b}\cos\frac{p\pi}{l}.
\end{align*}
Низший тип таких колебаний -- \(H_{101}\).

Рассмотрим теперь круглый резонатор:
\[
	k^2 = \left(\frac{\nu_{mn}}{a}\right)^2+
					 \left(\frac{p\pi}{l}\right)^2,
	\omega = \frac{k}{\sqrt{\eps\mu}},
\]

Для E-колебаний
\begin{align*}
	E_z & = E_0 J_m(\frac{\nu_{mn}}{a}r)\Phi(m\phi)\cos\frac{p\pi}{l},\\
	E_r & = -\frac{p\pi}{lg}E_0 J_m'(\frac{\nu_{mn}}{a}r)\Phi(m\phi)\sin\frac{p\pi}{l},\\
	E_\phi & = -\frac{p\pi}{l g^2}\frac{m}{r}E_0 J_m(\frac{\nu_{mn}}{a}r)\Phi'(m\phi)\sin\frac{p\pi}{l},\\
	H_r & = \frac{i\omega\eps}{g^2}\frac{m}{r}J_m(\frac{\nu_{mn}}{a}r)\Phi'(m\phi)\cos\frac{p\pi}{l},\\
	H_\phi & = -\frac{i\omega\eps}{g}J_m'(\frac{\nu_{mn}}{a}r)\Phi(m\phi)\cos\frac{p\pi}{l}.
\end{align*}

Для H-колебаний
\begin{align*}
	E_r & = -\frac{ip\pi\omega\mu}{g^2 l}\frac{m}{r}H_0J_m(\frac{\mu_{mn}}{a}r)\Phi'(m\phi)\sin\frac{p\pi}{l},\\
	E_\phi & = -\frac{ip\pi\omega\mu}{g l}H_0J_m'(\frac{\mu_{mn}}{a}r)\Phi(m\phi)\sin\frac{p\pi}{l},\\
	H_z & = H_0 J_m(\frac{\mu_{mn}}{a}r)\Phi(m\phi)\sin\frac{p\pi}{l},\\
	H_r & = -\frac{p\pi}{gl}H_0 J_m'(\frac{\mu_{mn}}{a}r)\Phi(m\phi)\cos\frac{p\pi}{l},\\
	H_\phi & = -\frac{p\pi}{g^2l}\frac{m}{r}H_0 J_m(\frac{\mu_{mn}}{a}r)\Phi'(m\phi)\cos\frac{p\pi}{l}.
\end{align*}
Было бы неплохо всё это вывести.

\section{Добротность объёмных резонаторов}

Начнём с баланса энергии
\[
	-\oint_S\vec{\Pi}\cdot\vec{dS} =
	i\omega\int_V\left(\frac{\eps\vec{E}\cdot\vec{E}^*}{2} -
    \frac{\mu^*\vec{H}\cdot\vec{H}^*}{2}\right)\,dV +
    \frac{1}{2}\int_V\vec{j}\cdot\vec{E}^*\,dV,
\]
Для поглощаемой мощности
\[
    -\Re\oint_S\vec{\Pi}\cdot\vec{dS} =
    \omega\int_V\left(\frac{\eps''\vec{E}\cdot\vec{E}^*}{2} +
    \frac{\mu''\vec{H}\cdot\vec{H}^*}{2}\right)\,dV + \Re
    \frac{1}{2}\int_V\vec{j}\cdot\vec{E}^*\,dV,
\]
В левой части стоит мощность, излучения с обратным знаком:
\[
	\Re\oint_S\vec{\Pi}\cdot\vec{dS} = P_\text{и} = P_\Sigma + P_{me}.
\]
Для распространяющейся в резонаторе энергии
\[
    -\Im\oint_S\vec{\Pi}\cdot\vec{dS} = \omega\int_V\left(\frac{\eps'\vec{E}\cdot\vec{E}^*}{2} -
    \frac{\mu'\vec{H}\cdot\vec{H}^*}{2}\right)\,dV + \Im P_\text{вз}
\]
или
\[
    -\Im\oint_S\vec{\Pi}\cdot\vec{dS} = 2\omega\int_V\left(\average{w_E} - \average{w_H}\right)\,dV + \Im P_\text{вз}.
\]
Добротность
\[
	Q = \frac{2\pi W}{PT} = \frac{\omega W}{P},\quad \dot{W} = -\frac{\omega}{Q}W
\]
отсюда, энергия в резонаторе меняется по закону
\[
	W = W_0 e^{-\frac{\omega t}{Q}}.
\]
Для напряжённости поля
\[
	\vec{E} = \vec{E}(0)e^{-\frac{\omega t}{2Q}}e^{i\omega t}
\]
Для комплексной частоты
\[
	\omega = \omega' + i\omega'',\quad \omega' = \omega, \quad \omega'' = \frac{\omega}{2Q}.
\]
Удобнее рассмотреть
\[
	\frac{1}{Q} = \frac{P_\text{ср} + P_{me} + P_\Sigma + P_\text{вз}}{\omega W} = \frac{1}{Q_0} + \frac{1}{Q_{ex}}.
\]
При расчёте собственной добротности \(Q_0\) учитываются 3 фактора: диэлектрические, магнитные и металлические потери.
\[
	Q_{me} = 2\frac{\mu}{\mu_{me}}\sqrt{\frac{\omega\mu_{me}\sigma}{2}}
	\frac{\int_V H^2 dV}{\oint_S H_\tau^2 dS}.
\]

\begin{problem}
	Имеется достаточно длинный отрезок волновода 23 на 10. Из мего требуется сделат перестраиваемый резонатор. Определить диапазон частот перестройки и длину резонатора при невозможности возникновения высших гармоник.
\end{problem}

\[
	\omega = \pi \sqrt{\left(\frac{m}{a}\right)^2 + \left(\frac{n}{b}\right)^2 + \left(\frac{p}{l}\right)^2}.
\]

Низшим перестраиваемым типом типом будет \( H_{101} \). Чтобы оно было низшим, необходимо:
\[
	\omega_{101} < \omega_{110},\quad  l > b.
\]
Частота при этом перестраивается в диапазоне
\[
	c/2a < f < c/2\sqrt{1/a^2 + 1/b^2},
\]
\[
	6.5~\text{ГГц} < f < 16~\text{ГГц}.
\]
С другой стороны, низшая частота для \( H_{201} \) равна \( f_max = 13\text{ГГц} \). Оценим максимальный размер волновода:
\[
	l_{min} = 1/\sqrt{c^2f_{max}^2/4 - 1/a^2} = a/\sqrt{3} = 13\text{мм}.
\]
\[
	6.5~\text{ГГц} < f < 13~\text{ГГц}.
\]

\begin{problem}
	Определить отношение a/l, при котором частоты низших типов E и H колебаний совпадают.
\end{problem}

\[
	\mu_{11}^2/a^2 + \pi^2/l^2 = \nu_{01}^2/a^2, l/a = \pi/\sqrt{\nu_{01}^2 - \mu_{11}^2} = 2.
\]
\begin{problem}
Определить 5 низших частот и указать типы колебаний резонатора \(1\times2\times3\) см и для круглого a = 1см, l = 2,5 см.
\end{problem}
\begin{problem}
Определить размеры резонатора, у которого \(E_{010}\) 5 ГГц, \(H_{111}\) 3,5 ГГц.
\end{problem}
\begin{problem}
Определить добротность резонатора 34 72 100 работающего на основном типе колебания, если материал стенок -- медь, а внутреннее заполнение имеет \(\eps = 2,35\), \(\tg\delta_\eps = 1,3\cdot10^{-5}\).
\end{problem}

Основной тип колебаний \( H_{011} \) с частотой
\[
	\omega = \frac{c\pi}{\sqrt{\eps}}\sqrt{\frac{1}{b^2} + \frac{1}{l^2}}.
\]
\[
	Q_\eps = \frac{1}{\tg\delta_\eps},
\]
\[
	Q_{me} = 2\frac{\mu}{\mu_{me}}\sqrt{\frac{\omega\mu_{me}\sigma}{2}}
	\frac{\int_V H^2 dV}{\oint_S H_\tau^2 dS}.
\]
\[
	\int_V H^2 dV = H_0^2\left( 1 + \frac{b^2}{l^2} \right)\frac{abl}{4},
\]
\[
	\int_S H^2 dS = 2H_0^2\left( 1 + \frac{b^2}{l^2} \right)\frac{bl}{4} +
	2H_0^2\frac{al}{2} + 2H_0^2\frac{b^2}{l^2}\frac{ab}{2},
\]
\[
	Q_{me} = \frac{\mu}{\mu_{me}}\sqrt{\frac{\omega\mu_{me}\sigma}{2}}
	\frac{\left( 1 + \frac{b^2}{l^2} \right)\frac{abl}{4}}{\left( 1 + \frac{b^2}{l^2} \right)\frac{bl}{4} + \frac{al}{2} + \frac{b^2}{l^2}\frac{ab}{2}}.
\]

\chapter{Возбуждение волноводов и резонаторов}
Предположим, что в резонаторе задан сторонний ток \( \vec{j}^\text{ст} \), определяемый только внешним источником и не зависящий от поля в волноводе, а на поврхности есть окно связи, в котором заданы поля.

Обычно сторонний ток вводится либо электрическим зондом (штырь), либо магнитным (петля).

Граничные условия в этом случае
\begin{align*}
	& \vec{E}_\tau = 0 (S - S_0),\\
	& \vec{E}_\tau = \vec{E}_{\tau0} (S_0),\\
	& \vec{H}_\tau = \vec{H}_{\tau0} (S_0).\\
\end{align*}

\section{Возбуждение волновода}
Рассмотрим теперь волновод, в котором возбуждающие элементы сосредоточены на отрезке \([z_1, z_2]\). При этом в волноводе в обе стороны от этой области распространяются волны. Поля будем искать на достаточно большом расстоянии от источников. Поле будем искать в виде разложения по собственным волнам (\( z > z_2 \)):
\[
	\vec{E} = \sum A_m\vec{E}_m,\quad \vec{H} = \sum A_m\vec{H}_m.
\]

Лемма Лоренца:
\[
  \oint_S\left( \vec{E}_1\times\vec{H}_2 - \vec{E}_2\times\vec{H}_1 \right)\cdot d\vec{S} = \int_V \left(
  \vec{j}_1^\text{ст}\cdot\vec{E}_2 - \vec{j}_2^\text{ст}\cdot\vec{E}_1\right) dV.
\]

Рассмотрим теперь в качестве \( \vec{E}_1 \) поля в системе, а \( \vec{E}_2 \) -- поле некоторой собственной волны. Тогда
\[
  \oint_S\left( \vec{E}\times\vec{H}_m - \vec{E}_m\times\vec{H} \right)\cdot d\vec{S} = \int_V \vec{j}^\text{ст}\cdot\vec{E}_m dV.
\]

Распишем подробнее:
\begin{gather*}
  \int_{S_2}\left( \sum A_n\vec{E}_n\times\vec{H}_m - \vec{E}_m\times\sum A_n\vec{H}_n \right)\cdot\vec{z}_0 dS
  -\int_{S_2}\left( \sum A_{-n}\vec{E}_{-n}\times\vec{H}_m - \vec{E}_m\times\sum A_{-n}\vec{H}_{-n} \right)\cdot\vec{z}_0 dS +\\+
  \int_S\left( \vec{E}\times\vec{H}_m - \vec{E}_m\times\vec{H} \right)\cdot d\vec{S} = \int_V \vec{j}^\text{ст}\cdot\vec{E}_m dV.
\end{gather*}

Так как \( \vec{E}_S = 0 \),
\begin{gather*}
  \sum A_n\int_{S_2}\left(\vec{E}_n\times\vec{H}_m - \vec{E}_m\times\vec{H}_n \right)\cdot\vec{z}_0 dS
  -\sum A_{-n}\int_{S_2}\left( \vec{E}_{-n}\times\vec{H}_m - \vec{E}_m\times\vec{H}_{-n} \right)\cdot\vec{z}_0 dS +\\+
  \int_{S_0} \vec{E}_0\times\vec{H}_m \cdot d\vec{S} = \int_V \vec{j}^\text{ст}\cdot\vec{E}_m dV.
\end{gather*}

Учитывая ортогональность собственных волн для положительного \( m \) имеем

\[
  -A_{-m}\int_{S_2}\left( \vec{E}_{-m}\times\vec{H}_m - \vec{E}_m\times\vec{H}_{-m} \right)\cdot\vec{z}_0 dS =
  -\int_{S_0} \vec{E}_0\times\vec{H}_m \cdot d\vec{S} + \int_V \vec{j}^\text{ст}\cdot\vec{E}_m dV.
\]

Рассмотрим норму волны
\[
	N_m = \int \vec{E}_m \times \vec{H}_m^* \cdot d\vec{S}
\]

\[
	A_{-m} = \frac{1}{2N_m} \left[ -\int_{S_0} \vec{E}_0\times\vec{H}_m \cdot d\vec{S} + \int_V \vec{j}^\text{ст}\cdot\vec{E}_m dV \right]
\]

Для \( -m \) гармоники имеем

\[
  A_{m}\int_{S_2}\left( \vec{E}_{m}\times\vec{H}_{-m} - \vec{E}_{-m}\times\vec{H}_{m} \right)\cdot\vec{z}_0 dS =
  -\int_{S_0} \vec{E}_0\times\vec{H}_{-m} \cdot d\vec{S} + \int_V \vec{j}^\text{ст}\cdot\vec{E}_{-m} dV.
\]

\[
	A_m = \frac{1}{2N_m} \left[ -\int_{S_0} \vec{E}_0\times\vec{H}_{-m} \cdot d\vec{S} + \int_V \vec{j}^\text{ст}\cdot\vec{E}_{-m} dV \right]
\]

Итого:

\[
	A_{\pm m} = \frac{1}{2N_m} \left[ \int_V \vec{j}^\text{ст}\cdot\vec{E}_{\mp m} dV - \int_{S_0} \vec{E}_0\times\vec{H}_{\mp m} \cdot d\vec{S} \right]
\]

\section{Возбуждение резонаторов}
\[
	\vec{E} = \sum A_n \vec{E}_n,\quad \vec{H} = \sum B_n \vec{H}_n
\]
Запишем уравнения Максвелла для собственных колебаний
\[
	\begin{cases}
	\rotor \vec{E}_n = -i\omega_n\mu\vec{H}_n\\
	\rotor \vec{H}_n = i\omega_n\eps\vec{E}_n,
	\end{cases}
\]
и для картины поля вцелом
\[
	\begin{cases}
	\rotor \vec{E} = -i\omega\mu\vec{H}\\
	\rotor \vec{H} = i\omega\eps\vec{E} + \vec{j}^\text{ст}.
	\end{cases}
\]

\[
	\divergence \vec{E}_n\times\vec{H}^* = \vec{H}^* \cdot \rotor\vec{E}_n -\vec{E}_n \cdot \rotor\vec{H}^* = -i\omega_n\mu\vec{H}_n\cdot\vec{H}^* + i\omega\eps^*\vec{E}_n\vec{E}^* - {\vec{j}^\text{ст}}^*\vec{E}_n
\]

\[
	\divergence \vec{E}\times\vec{H}_n^* = \vec{H}_n^* \cdot \rotor\vec{E} -\vec{E} \cdot \rotor\vec{H}_n^* = -i\omega\mu\vec{H}\cdot\vec{H}_n^* + i\omega_n\eps^*\vec{E}\vec{E}_n^*.
\]

Интегрируя два этих выражения по объёму резонатора получаем

\[
	\oint \vec{E}_n\times\vec{H}^* \cdot d\vec{S} = 0 = -i\omega_n 2W_nB_n^* + i\omega 2W_nA_n^* + \int {\vec{j}^\text{ст}}^*\vec{E}_n dV,
\]

\[
	\oint \vec{E}\times\vec{H}_n^* \cdot d\vec{S} = \int_{S_0} \vec{E}_0\times\vec{H}_n^* \cdot d\vec{S} = -i\omega 2W_nB_n + i\omega_n 2W_nA_n.
\]

Получаем систему из двух уравнений с двумя неизвестными
\[
	\begin{cases}
		\omega A_n - \omega_n B_n = \cfrac{i}{2W_n}\int \vec{j}^\text{ст}\vec{E}_n^* dV\\
		\omega_n A_n - \omega B_n = -\cfrac{i}{2W_n} \int_{S_0} \vec{E}_0\times\vec{H}_n^* \cdot d\vec{S}.
	\end{cases}
\]
Отсюда
\begin{align*}
	& A_n = \frac{1}{2W_n(\omega^2-\omega_n^2)}\left[ \omega\int \vec{j}^\text{ст}\vec{E}_n^* dV + \omega_n\int_{S_0} \vec{E}_0\times\vec{H}_n^* \cdot d\vec{S} \right],\\
	& B_n = -\frac{1}{2W_n(\omega^2-\omega_n^2)}\left[ \omega_n\int \vec{j}^\text{ст}\vec{E}_n^* dV + \omega\int_{S_0} \vec{E}_0\times\vec{H}_n^* \cdot d\vec{S} \right].
\end{align*}

\end{document}
